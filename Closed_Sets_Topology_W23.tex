\achapter{13}{Closed Sets in Topological Spaces}\label{sec:Closed_sets_topology}


\vspace*{-17 pt}
\framebox{
\parbox{\dimexpr\linewidth-3\fboxsep-3\fboxrule}
{\begin{fqs}
\item What does it mean for a set to be closed in a topological space?
\item What important properties do closed sets have in relation to unions and intersections?
\item What is a sequence in a topological space?
\item What does it mean for a sequence to converge in a topological space?
\item What is a limit of a sequence in a topological space?
\item What is a limit point of a subset of a topological space? How are closed sets related to limit points? 
\item What is a boundary point of a subset of a topological space and what is the boundary of a subset of a topological space? How are closed sets related to boundary points? 
\item What does it mean for a space to be Hausdorff? What important properties do Hausdorff spaces have?
\item What are the separation axioms $T_1$, $T_2$, $T_3$, and $T_4$. What is the underlying idea behind these properties?
\end{fqs}}}

\vspace*{13 pt}

\csection{Introduction}
We defined a closed set in a metric space to be the complement of an open set. Since a topology is defined in terms of open sets, we can make the same definition of closed set in a topological space. With the definition of closed set in hand, we can then ask if it is possible to define limit points, boundary, and closure in topological spaces and determine if there are corresponding properties for these ideas in topological spaces. 

\begin{definition} A subset $C$ of a topological space $X$ is \textbf{closed}\index{closed set in a topological space} if its complement $X \setminus C$ is open. 
\end{definition}

\begin{pa} ~
\be
\item List all of the closed sets in the indicated topological space.
	\ba
	\item $(X, \tau)$ with $X= \{a,b,c,d\}$ and $\tau = \{\emptyset, \{a\}, \{b\}, \{a,b\}, X \}$.

	\item $(X, \tau)$ with $X= \{a,b,c,d,e,f\}$ and $\tau = \{\emptyset,\{a\}, \{c,d\}, \{a,c,d\}, \{b,c,d,e,f\}, X\}$.

	\item $(X, \tau)$ with $X = \R$ and $\tau = \{\emptyset, \{0\}, \R\}$. 

	\item $(X, \tau)$ with $X = \{a,b,c\}$ and $\tau = \{\emptyset, \{a\}, \{b\},\{c\}, \{a,b\}, \{a,c\}, \{b,c\}, X \}$. (What is the name of this topology?)

	\item $(X, \tau)$ with $X=\Z^+$ and $\tau = \{\emptyset, X\}$ (this topology is called the \emph{indiscrete} or \emph{trivial} topology). 

	\ea


\item Using the examples from part (1), find (if possible), a set that is
	\ba
	\item both closed and open (if possible, find one that is not the entire set or the empty set)
	
	\item closed but not open
	
	\item open but not closed
	
	\item not open and not closed

	\ea

\item In $\R^n$ with the Euclidean metric, every single element set is closed. Does this property hold in the topological space $(X, \tau)$, where $X = \{a, b, c\}$ and $\tau = \{\emptyset, \{a\}, \{a, b\}, \{a, c\}, X\}$? Explain. 


\ee

\end{pa}

\begin{comment}

\ActivitySolution

\be
\item List all of the closed sets in the indicated topological space.
	\ba
	\item  The closed sets are the complements of the open sets, so the closed sets are 
\[X, \{b,c,d\}, \{a,c,d\}, \{c,d\}, \text{ and } \emptyset.\]

	\item  The closed sets are the complements of the open sets, so the closed sets are 
\[X, \{b,c,d,e,f\}, \{a,b,e,f\}, \{b,e,f\}, \{a\}, \text{ and } \emptyset.\]


	\item The closed sets are the complements of the open sets, so the closed sets are 
\[\R, \R-\{0\}, \text{ and } \emptyset.\]

	\item  Every subset is open, so this topology is the discrete topology. The closed sets are the complements of the open sets, so every subset is also closed. So the closed sets are 
\[X, \{b,c\}, \{a,c\}, \{a,b\}, \{c\}, \{b\},\{a\}, \text{ and } \emptyset.\]

	\item  The closed sets are the complements of the open sets, so the closed sets are 
\[X  \text{ and } \emptyset.\]

	\ea


\item 
	\ba
	\item  In the topological space $(X, \tau)$ with $X= \{a,b,c,d,e,f\}$ and $\tau = \{\emptyset,\{a\}, \{c,d\}, \{a,c,d\}, \{b,c,d,e,f\}, X\}$ the set $\{a\}$ is both open and closed. 
	
	\item  In the topological space $(X, \tau)$ with $X= \{a,b,c,d\}$ and $\tau = \{\emptyset, \{a\}, \{b\}, \{a,b\}, X \}$ the set $\{a,c,d\}$ is closed but not open. 
	
	\item  In the topological space $(X, \tau)$ with $X= \{a,b,c,d\}$ and $\tau = \{\emptyset, \{a\}, \{b\}, \{a,b\}, X \}$ the set $\{a,b\}$ is open but not closed. 
	
	\item  Consider the topological space $(X, \tau)$ with $X= \{a,b,c,d\}$ and $\tau = \{\emptyset, \{a\}, \{b\}, \{a,b\}, X \}$. The set $\{b,c\}$ is neither open nor closed. 
	
	\ea

\item Since $X \setminus \{a\} = \{b,c\}$ and $\{b,c\}$ is not open, the single element set $\{a\}$ is not closed. 

\ee

\end{comment}

\csection{Unions and Intersections of Closed Sets}

Now we have defined open and closed sets in topological spaces. In our preview activity we saw that a set can be both open and closed. As we did in metric spaces, we will call any set that is both open and closed a \emph{clopen}\index{clopen set in a topological space} (for closed-open) set.

By definition, any union and any finite intersection of open sets in a topological space is open, so the fact that closed sets are complements of open sets implies the following theorem. 

\begin{theorem} \label{thm:closed_TS} Let $X$ be a topological space.
\begin{enumerate}
\item Any intersection of closed sets in $X$ is a closed set in $X$.
\item Any finite union of closed sets in $X$ is a closed set in $X$. 
\end{enumerate}
\end{theorem}

\begin{proof} Let $X$ be a topological space. To prove part 1, assume that $C_{\alpha}$ is a collection of closed set in $X$ for $\alpha$ in some indexing set $I$. Then 
\[X \setminus \bigcap_{\alpha \in I} C_{\alpha} = \bigcup_{\alpha \in I} X \setminus C_{\alpha}.\]
The latter is an arbitrary union of open sets and so it an open set. By definition, then, $\bigcap_{\alpha \in I} C_{\alpha}$ is a closed set. 

For part 2, assume that $C_1$, $C_2$, $\ldots$, $C_n$ are closed sets in $X$ for some $n \in \Z^+$. To show that $C = \bigcap_{k=1}^n C_k$ is a closed set, we will show that $X \setminus C$ is an open set. Now 
\[X \setminus \bigcup_{\alpha \in I} C_{\alpha} = \bigcap_{\alpha \in I} X \setminus C_{\alpha}\]
is a finite intersection of open sets, and so is an open set. Therefore, $\bigcup_{\alpha \in I} C_{\alpha} $ is a closed set. 
\end{proof}

Theorem \ref{thm:closed_TS} tells us that any intersection of closed sets is closed, but only finite unions of closed sets are closed. How do we know that non-finite unions of closed sets aren't necessarily closed?

\begin{activity} Let $\Z$ be a topological space with the finite complement topology $\tau_{FC}$. That is, a non-empty set $O$ is open in $\Z$ if $\Z \setminus O$ is finite. 
	\ba
	\item What must be true about the cardinality of the closed sets in $(\Z, \tau_{FC})$? 
	
	\item Let $C_n = \{2, 3, \ldots, n\}$. Is the set $\bigcup_{n \geq 3} C_n$ a closed set in $(\Z, \tau_{FC})$? Explain. 

	\ea
	
\end{activity}
	
\begin{comment}

\ActivitySolution

\ba

\item If $C$ is closed, then $C$ is the complement of some open set $O$. That is, $C = \Z \setminus O$ is finite. 

\item Since $\bigcup_{n \geq 3} C_n = \{n \in \Z \mid n \geq 2\}$ is an infinite set, we can see that the union of all of the $C_n$ is not a closed set in $(\Z, \tau_{FC})$. We conclude that arbitrary unions of closed sets need not be closed. 

\ea

\end{comment}

\csection{Limit Points and Sequences in Topological Spaces}

Recall that we defined a limit point of a set $A$ in a metric space $X$ to be a point $x \in X$ such that every neighborhood of $x$ contains a point in $A$ different from $x$. Since we have defined neighborhoods in topological spaces, we can make the same definition. 

\begin{definition} Let $X$ be a topological space, and let $A$ be a subset of $X$. A \textbf{limit point}\index{limit point in a topological space} of $A$ is a point $x \in X$ such that every neighborhood of $x$ contains a point in $A$ different from $x$. 
\end{definition}

The set $A'$ of limit points of $A$ is called the \emph{derived set}\index{derived set} of $A$. 

\begin{activity} Find the limit point(s) of the following sets
\ba
\item $\{c,d\}$ in $(X, \tau)$ with $X= \{a,b,c,d\}$ and $\tau = \{\emptyset, \{a\}, \{b\}, \{a,b\}, X \}$

\item $\{a,b\}$ in the set $X= \{a,b,c,d,e,f\}$ with topology 
\[\tau= \{\emptyset,\{b\}, \{a,b,c\},\{d,e,f\},\{b,d,e,f\}, X\}.\] 

\item $\{a,b\} \subset X$ where $X = \{a,b,c\}$ in the discrete topology. 

\item $\{-1,0,1\} \subset \Z$ with $\tau$ the topology on $\Z$ with basis $\{B(n)\}$, where 
\[B(n) = \begin{cases} \{n\}	&\text{if $n$ is odd}, \\ \{n-1,n,n+1\}	&\text{if $n$ is even}. \end{cases}\]
(This topology is called the \emph{digital line topology} and has applications in digital processing. (This topology is called the \emph{digital line topology} and has applications in digital processing. That the collection $\{B(n)\}$ is a basis for a topology on $\Z$ is the topic of Exercise (\ref{ex:digital_line_topology}) on page \pageref{ex:digital_line_topology}.)
 
 \ea

\end{activity}

\begin{comment}

\ActivitySolution

\ba
\item  Neither $a$ nor $b$ is a limit point of $\{c,d\}$ since the open neighborhood $\{a,b\}$ contains no point in $\{c,d\}$ different than $a$ or $b$. The only open set that contains $c$ or $d$ is $X$, so that is the only neighborhood of $c$ or $d$. Since $X$ contains a point in $\{c,d\}$ that is different than $c$ (or $d$), both $c$ and $d$ are limit points of $\{c,d\}$. 

\item  None of the points $b$, $d$, $e$, or $f$ is a limit point of $\{a,b\}$ since the open neighborhood $\{b,d,e,f\}$ contains no point in $\{a,b\}$ different than $b$, $d$, $e$, or $f$. Any neighborhood of $a$ or $c$ must contain one of the open sets $\{a,b,c\}$ or $X$. So every neighborhood of $a$ or $c$ contains a point of $\{a,b\}$ different than $a$ or $c$. Therefore, the limit points of $\{a,b\}$ are $a$ and $c$ and $\{a,b\}' = \{a,c\}$. 

\item  For any $x \in \{a,b\}$, the open neighborhood $\{x\}$ of $x$ does not contain any points in $\{a,b\}$ different than $x$. So the set $\{a,b\}$ has no limit points. 

\item  Any neighborhood of $0$ must contain $B(0) = \{-1,0,1\}$, and so every neighborhood of $0$ contains a point in $\{-1,0,1\}$ different from $0$. Similarly, any neighborhood of $2$ must contain $B(2) = \{1,2,3\}$, and so every neighborhood of $2$ contains a point in $\{-1,0,1\}$ different from $2$. Also, any neighborhood of $-2$ must contain $B(-2) = \{-3,-2,-1\}$, and so every neighborhood of $-2$ contains a point in $\{-1,0,1\}$ different from $-2$. Thus, $\{-2,0,2\} \subseteq \{-1,0,1\}'$. If $n$ is odd, then the open neighborhood $B(n) = \{n\}$ contains no points of $\{-1,0,1\}$ different than $n$. So no odd integer is a limit point of $\{-1,0,1\}$. If $n$ is an even integer different than $-2$, $0$, and $2$, then the open neighborhood $B(n) = \{n-1,n,n+1\}$ contains no points in $\{-1,0,1\}$. Therefore, $\{-1,0,1\}' = \{-2,0,2\}$. 

\ea

\end{comment}

In metric spaces, a set is closed if and only if it contains all of its limit points. So the corresponding result in topological spaces should be no surprise. 

\begin{theorem} \label{thm:TS_closed_limitpoints} Let $C$ be a subset of a topological space $X$, and let $C'$ be the set of limit points of $C$. Then $C$ is closed if and only if $C' \subseteq C$.  
\end{theorem}

\begin{proof} Let $X$ be a topological space, and let $C$ be a subset of $X$. First we assume that $C$ is closed and show that $C$ contains all of its limit points. Let $x \in X$ be a limit point of $C$. We proceed by contradiction and assume that $x \notin C$. Then $x \in X \setminus C$, which is an open set. This means that there is a neighborhood (namely $X \setminus C$) of $x$ that contains no points in $C$, which contradicts the fact that $x$ is a limit point of $C$. We conclude that $x \in C$ and $C$ contains all of its limit points.

For the converse, assume that $C$ contains all of its limit points. To show that $C$ is closed, we prove that $X \setminus C$ is open. We again proceed by contradiction and assume that $X \setminus C$ is not open. Then there exists $x \in X \setminus C$ such that no neighborhood of $x$ is entirely contained in $X \setminus C$. This implies that every neighborhood of $x$ contains a point in $C$ and so $x$ is a limit point of $C$. It follows that $x \in C$, contradicting the fact that $x \in X \setminus C$. We conclude that $X \setminus C$ is open and $C$ is closed.
\end{proof}

In metric spaces we saw that limit point of a set is the limit of a sequence of points in the set. To explore this idea in topological spaces, we define sequences in the same way we did in metric spaces. 

\begin{definition} A \textbf{sequence}\index{sequence in a topological space} in a topological space $X$ is a function $f: \Z^+$ to $X$.
\end{definition}

We use the same notation and terminology related to sequences as we did in metric spaces: we will write $(x_n)$ to represent a sequence $f$, where $x_n = f(n)$ for each $ n \in \Z^+$. We can't define convergence in a topological space using a metric, but we can use open sets. Recall that a sequence $(x_n)$ in a metric space $(X,d)$ converges to a point $x$ in the space if, given $\epsilon > 0$ there exists a positive integer $N$ such that $d(x_n,x) < \epsilon$ for all $n \geq N$. In other words,  every open ball centered at $x$ contains all of the entries of the sequence past a certain point. We can replace open balls with open sets and make a similar definition of convergence in topological spaces. 

\begin{definition} A sequence $(x_n)$ in a topological space $X$ \textbf{converges}\index{convergent sequence in a topological space} to the point $x \in X$ if, for each open set $O$ that contains $x$ there exists a positive integer $N$ such that $x_n \in O$ for all $n \geq N$. 
\end{definition}

If a sequence $(x_n)$ converges to a point $x$, we call $x$ a \emph{limit}\index{limit of a sequence in a topological space} of the sequence $(x_n)$. 

\begin{activity} \label{act:TS_limits} In metric spaces, limits of sequences are unique. We may wonder if the same result is true in topological spaces. Consider the topological space $(X,\tau)$, where $X = \{a, b, c\}$ and $\tau = \{\emptyset, \{c\}, \{a, c\}, \{b, c\}, X\}$. Find all limits of all constant sequences in $X$.

\end{activity}

\begin{comment}

\ActivitySolution We start by finding all neighborhoods. 
\begin{itemize}
\item The neighborhoods of $a$ are $\{a,c\}$ and $X$.
\item The neighborhoods of $b$ are $\{b,c\}$ and $X$.
\item The neighborhoods of $c$ are $\{c\}$, $\{a,c\}$, $\{b,c\}$, and $X$.
\end{itemize}

Consider the sequence $(a)$. The neighborhood $\{b,c\}$ of $b$ does not contain $a$, so $b$ is not a limit of the sequence $(a)$. The neighborhood $\{c\}$ of $c$ does not contain $a$, so $c$ is not a limit of the sequence $(a)$. Therefore, the only limit of the sequence $(a)$ is $a$.

Now consider the sequence $(b)$. The neighborhood $\{a,c\}$ of $a$ does not contain $b$, so $a$ is not a limit of the sequence $(a)$. The neighborhood $\{c\}$ of $c$ does not contain $b$, so $c$ is not a limit of the sequence $(b)$. Therefore, the only limit of the sequence $(b)$ is $b$.

Finally, consider the sequence $(c)$.  Each neighborhood of $a$ and $b$ contains the entire sequence $(c)$, so $a$ and $b$ are both limits of $(c)$. It follows that every point in $X$ is a limit of the sequence $(a)$.

\end{comment}


The result of Activity \ref{act:TS_limits} is that sequences do not behave in topological spaces as we would expect them to. Consequently, sequences do not play the same important role in topological spaces as they do in metric spaces. However, the concept of limit point is important, as are the notions of boundary and closure in topological spaces. 

\csection{Closure in Topological Spaces}

Once we have a definition of limit point, we can define the closure of a set just as we did in metric spaces. 

\begin{definition} The \textbf{closure}\index{closure in topological spaces} of a subset $A$ of a topological space $X$ is the set 
\[\overline{A} = A \cup A'.\]
\end{definition}

In other words, the closure of a set is the collection of the elements of the set and the limit points of the set. The following theorem is the analog of the theorem in metric spaces about closures. 

\begin{theorem} \label{thm:TS_closure_closed} Let $X$ be a topological space and $A$ a subset of $X$. The closure of a $A$ is a closed set. Moreover, the closure of $A$ is the smallest closed subset of $X$ that contains $A$. 
\end{theorem}

\begin{proof} Let $X$ be a topological space and $A$ a subset of $X$. To prove that $\overline{A}$ is a closed set, we will prove that $\overline{A}$ contains its limit points. Let $x \in \overline{A}'$. To show that $x \in \overline{A}$, we proceed by contradiction and assume that $x \notin \overline{A}$. This implies that $x \notin A$ and $x \notin A'$. Since $x \notin A'$, there exists a neighborhood $N$ of $x$ that contains no points of $A$ other than $x$. But $A \subseteq \overline{A}$ and $x \notin \overline{A}$, so it follows that $N \cap A = \emptyset$. This implies that there is an open set $O \subseteq N$ centered at $x$ so that $O \cap A = \emptyset$. The fact that $x \in \overline{A}'$ means that $O \cap \overline{A}$ contains a point $y$ in $\overline{A}$ different from $x$. Since $O \cap A = \emptyset$, we must have $y \in A'$. But the fact that $O$ is a neighborhood of $y$ means that $O$ must contain a point of $A$ different than $y$, which contradicts the fact that $O \cap A = \emptyset$. We conclude that $x \in \overline{A}$ and $\overline{A}' \subseteq \overline{A}$. This shows that $\overline{A}$ is a closed set. 

The proof that $\overline{A}$ is the smallest closed subset of $X$ that contains $A$ is left for the next activity.
\end{proof}

\begin{activity} Let $(X,d)$ be a topological space, and let $A$ be a subset of $X$. 
\ba
\item What will we have to show to prove that $\overline{A}$ is the smallest closed subset of $X$ that contains $A$?

\item Suppose that $C$ is a closed subset of $X$ that contains $A$. To show that $\overline{A} \subseteq C$, why is it enough to demonstrate that $A' \subseteq C$? 

\item If $x \in A'$, what can we say about $x$? 

\item Complete the proof that $\overline{A} \subseteq C$.

\ea

\end{activity}

\begin{comment}

\ActivitySolution

\ba
\item We need to prove that any closed subset of $X$ that contains $A$ also contains $\overline{A}$.

\item Since $\overline{A} = A \cup A'$, if $C$ already contains $A$, then to show that $\overline{A} \subseteq C$, we only need to show that $A' \subseteq C$. 

\item If $x \in A'$, then $x$ is a limit point of $A$. That means that every neighborhood of $x$ in $X$ contains a point in $A$ different from $x$.

\item Let $x \in A'$, and let $N$ be a neighborhood of $x$. Then $N$ contains a point of $A$ different than $x$. Since $A \subseteq C$, it follows that $N$ contains a point of $C$ different than $x$. So $x$ is a limit point of $C$. The fact that $C$ is closed means that $C$ contains its limit points, so $x \in C$. Therefore, $A' \subseteq C$ and $\overline{A} \subseteq C$. 

\ea

\end{comment}

One consequence of Theorem \ref{thm:TS_closure_closed} is the following.

\begin{corollary} A subset $C$ of a topological space $X$ is closed if and only if $C = \overline{C}$. 
\end{corollary}

\csection{The Boundary of a Set}

In addition to limit points, we also defined boundary points in metric spaces. Recall that a boundary point of a set $A$ in a metric space $X$ could be considered to be any point in $\overline{A} \cap \overline{X \setminus A}$. We make the same definition in a topological space. 

\begin{definition} Let $(X, \tau)$ be a topological space, and let $A$ be a subset of $X$. A \textbf{boundary point}\index{boundary point in a topological space} of $A$ is a point $x \in X$ such that every neighborhood of $x$ contains a point in $A$ and a point in $X \setminus A$. The \textbf{boundary}\index{boundary in a topological space} of $A$ is the set 
\[\Bdry(A) = \{x \in X \mid x \text{ is a boundary point of } A\}.\]
\end{definition}

As with metric spaces, the boundary points of a set $A$ are those points that are ``between" $A$ and its complement. 

\begin{activity} \label{act:TS_bl_examples} Find the boundaries of the following sets
\ba
\item $\{c,d\}$ in $(X, \tau)$ with $X= \{a,b,c,d\}$ and $\tau = \{\emptyset, \{a\}, \{b\}, \{a,b\}, X \}$.

\item $\{a,b\}$ in the set $X= \{a,b,c,d,e,f\}$ with topology 
\[\tau= \{\emptyset,\{b\}, \{a,b,c\},\{d,e,f\},\{b,d,e,f\}, X\}.\] 

\item $\{a,b\} \subset X$ where $X = \{a,b,c\}$ in the discrete topology.

\item $\Z$ in $\R$ with the finite complement topology $\tau_{FC}$. 

\ea

\end{activity}

\begin{comment}

\ActivitySolution

\ba
\item  Neither $a$ nor $b$ is a boundary point of $\{c,d\}$ since the open neighborhood $\{a,b\}$ contains no point in $\{c,d\}$. The only open set that contains $c$ or $d$ is $X$, so that is the only neighborhood of $c$ or $d$. Since $X$ contains a point in $\{c,d\}$ that is different than $c$ (or $d$), and a point not in $\{c,d\}$, both $c$ and $d$ are boundary points of $\{c,d\}$. Therefore, $\Bdry(\{c,d\}) = \{c,d\}$. 

\item  None of the points $d$, $e$, or $f$ is a boundary point of $\{a,b\}$ since the open neighborhood $\{d,e,f\}$ contains no point in $\{a,b\}$. The open neighborhood $\{b\}$ of $b$ contains no points that are not in $\{a,b\}$, so $b$ is not a boundary point of $\{a,b\}$. Any neighborhood of $a$ or $c$ must contain one of the open sets $\{a,b,c\}$ or $X$. So every neighborhood of $a$ or $c$ contains a point of $\{a,b\}$ and a point not in $\{a,b\}$. So $a$ and $c$ are boundary points of $\{a,b\}$. Therefore, $\Bdry(\{a,b\}) = \{a,c\}$.  

\item Since $\{a\}$, $\{b\}$, and $\{c\}$ are open sets, and none of these sets contain points in both $A$ and $X \setminus A$, the boundary of $A$ is empty. 

\item Let $x \in \R$ and let $O$ be an open set containing $x$. Since $\R \setminus O$ is finite, there must be infinitely many points in both $\Z$ and $\R$ that are not in $\R \setminus O$. So $O$ contains infinitely many integers and real numbers. It follows that every point in $\Z$ is a boundary point and $\Bdry(\Z) = \R$. 

\ea

\end{comment}

Just as with metric spaces, we can characterize the closed sets as the sets that contain their boundary.

\begin{theorem} \label{thm:TS_Closed_boundary} A subset $C$ of a topological space $X$ is closed if and only if $C$ contains its boundary. 
\end{theorem}

The proof of Theorem \ref{thm:TS_Closed_boundary} is left to Exercise (\ref{ex:TS_Closed_boundary}).

\csection{Separation Axioms}

As we have seen, sequences in topological spaces do not generally behave as we would expect them to. As a result, we look for conditions on topological spaces under which sequences do exhibit some regular behavior. In our preview activity we saw that in it is possible in a topological space that single point sets do not have to be closed. In Activity \ref{act:TS_limits}, we also saw that limits of sequences in topological spaces are not necessarily unique. This type of behavior limits the results that one can prove about such spaces. As a result, we define classes of topological spaces whose behaviors are closer to what our intuition suggests. 

\begin{activity} \label{act:Hausdorff} ~
\ba
\item Consider a metric space $(X,d)$, and let $x$ and $y$ be distinct points in $X$. 
	\begin{enumerate}[i.]
	\item Explain why $x$ and $y$ cannot both be limits of the same sequence if we can find disjoint open balls $B(x,r)$ centered at $x$ and $B(y,s)$ centered at $y$ such that $B_x \cap B_y = \emptyset$.
	
	\item Now show that we can find disjoint open balls $B(x,r)$ centered at $x$ and $B(y,s)$ centered at $y$ such that $B(x,s) \cap B(y,r) = \emptyset$.
	
	\end{enumerate}

\item Return to our example from Activity \ref{act:TS_limits} with $X = \{a, b, c\}$ and topology 
\[\tau~=~\{\emptyset, \{c\}, \{a, c\}, \{b, c\}, X\}.\] 
We saw that every point in $X$ is a limit of the constant sequence $(c)$. If $x \neq c$ in $X$, Explain why there are no disjoint open sets $O_x$ containing $x$ and $O_c$ containing $c$. 	

\ea

\end{activity}

\begin{comment}

\ActivitySolution

\ba
\item Consider a metric space $(x,d)$, and let $x$ and $y$ be distinct points in $X$. 
	\begin{enumerate}[i.]
	\item Suppose a sequence $(x_n)$ has $x$ as a limit. Then there exists a positive integer $N$ such that $n \geq n$ implies $x_n \in B(x,r)$. But then $x_n \notin B(y,s)$ and so $y$ cannot be a limit of the sequence $(x_n)$. 
	
	\item Let $r = \frac{d(x,y)}{2}$. To show that $B(x,r) \cap B(y,r) = \emptyset$, suppose $z \in (B(x,r) \cap B(y,r)$. Then $z \in B(x,r)$. The triangle inequality $d(x,y) \leq d(x,z) + d(z,y)$ shows that 
	\[d(z,y) \geq d(x,y) - d(x,z) > d(x,y) - r = 2r -r = r.\]
	But this contradicts $z \in B(y,r)$. We conclude that $B(x,r) \cap B(y,r) = \emptyset$.
	
	\end{enumerate}

\item The only open sets that contain $a$ are $\{a,c\}$ and $X$ and the only open sets that contain $b$ are $\{b,c\}$ and $X$. These open sets all contain $c$, so there are no disjoint open sets $O_x$ and $O_c$ containing $x$ and $c$. 	

\ea

\end{comment}

It is the fact as described in Activity \ref{act:Hausdorff} that we can separate disjoint points by disjoint open sets that separates metric spaces from other spaces where limits are not unique. If we restrict ourselves to spaces where we can separate points like this, then we might expect to have unique limits. Such spaces are called \emph{Hausdorff} spaces. 

\begin{definition} A topological space $X$ is a \textbf{Hausdorff}\index{Hausdoff space} space if for each pair $x$, $y$ of distinct points in $X$, there exists open sets $O_x$ of $x$ and $O_y$ of $y$ such that $O_x \cap O_y = \emptyset$. 
\end{definition}

Activity \ref{act:Hausdorff} shows that every metric space is a Hausdorff space. Once we start imposing conditions on topological spaces, we restrict the number of spaces we consider.

\begin{activity} ~
\ba
\item Let $X$ be any set and $\tau$ the discrete topology. Is $(X, \tau)$ Hausdorff? Justify your answer.

\item Let $(X, \tau)$ be a Hausdorff topological space with $X = \{x, x_1, x_2, \ldots, x_n\}$ a finite set. Let $x \in X$. Is $\{x\}$ an open set? Explain. What does this say about the topology $\tau$? (Hint: Is $x$ a limit point of $\{x_1, x_2, \ldots, x_n\}$?)

\ea

\end{activity}

\begin{comment}

\ActivitySolution

\ba
\item Let $x$ and $y$ be distinct elements in $X$. Since every subset of $X$ is open, the disjoint open set $\{x\}$ and $\{y\}$ separate $x$ and $y$. So $(X, \tau)$ is Hausdorff.

\item Suppose $X = \{x, x_1, x_2, \ldots, x_n\}$. Let $A = \{x_1, x_2, \ldots, x_n\}$. We will show that $A$ is closed by demonstrating that $A$ contains all of its limit points. To do this we only need to show that $x$ is not a limit point of $A$. Since $X$ is Hausdorff, for each $i$ there exists an open sets $Ox_i$ and $O_i$ such that $Ox_i \cap O_i = \emptyset$. Let $O = \cap Ox_i$. Then $O$ is neighborhood of $x$ and $O \cap O_i = \emptyset$ for every $i$. Thus, $O$ does not contain any points in $A$. Since $A$ is closed, it follows that $\{x\} = X \setminus A$ is open. Since every single element set is open, the topology $\tau$ is the discrete topology. 

\ea

\end{comment}

\begin{example} \label{exp:K_topology} There are examples of Hausdorff spaces that are not the standard metric spaces. For example, Let  $K = \left\{\frac{1}{k} \mid k \text{ is a positive integer} \right\}$. We use $K$ to make a topology on $\R$ with basis all open intervals of the form $(a,b)$ and all sets of the form $(a,b) \setminus K$, where $a < b$ are real numbers. This topology adds the extra intervals of the form $(a,b) \setminus K$ to the standard open intervals to make a new topology. This topology is known as the $K$-topology on $\R$. Just as in $(\R, d_E)$, if $x$ and $y$ are distinct real numbers we can separate $x$ and $y$ with open intervals. 


\end{example}

The reason we defined Hausdorff spaces is because they have familiar properties, as the next theorems illustrate. 

\begin{theorem} Each single point subset of a Hausdorff topological space is closed.
\end{theorem}

\begin{proof} Let $X$ be a Hausdorff topological space, and let $A = \{a\}$ for some $a \in X$. To show that $A$ is closed, we prove that $X \setminus A$ is open. Let $x \in X \setminus A$. Then $x \neq a$. So there exist open sets $O_x$ containing $x$ and $O_a$ containing $a$ such that $O_x \cap O_a = \emptyset$. So $a \notin O_x$ and $O_x \subseteq X \setminus A$. Thus, every point of $X \setminus A$ is an interior point and $X \setminus A$ is an open set. This verifies that $A$ is a closed set.
\end{proof}


\begin{theorem} A sequence of points in a Hausdorff topological space can have at most one limit in the space.
\end{theorem}

\begin{proof} Let $X$ be a Hausdorff topological space, and let $(x_n)$ be a sequence in $X$. Suppose $(x_n)$ converges to $a \in X$ and to $b \in X$. Suppose $a \neq b$. Then there exist open sets $O_a$ of $a$ and $O_b$ of $b$ such that $O_a \cap O_b = \emptyset$. But the fact that $(x_n)$ converges to $a$ implies that there is a positive integer $N$ such that $x_n \in O_a$ for all $n \geq N$. But then $x_n \notin O_b$ for any $n \geq N$. This contradicts the fact that $(x_n)$ converges to $b$. We conclude that $a=b$ and that the sequence $(x_n)$ can have at most one limit in $X$. 
\end{proof}

Hausdorff spaces are important because we can separate distinct points with disjoint open sets. It is also of interest to consider what other types of objects we can separate with disjoint open sets. For example, the indiscrete topology is quite bad in the sense that its open sets can't distinguish between distinct points. That is, if $x$ and $y$ are distinct points in a space with the indiscrete topology, then every open set that contains $x$ also contains $y$. By contrast, in a Hausdorff space we can separate distinct points with disjoint open sets. This is an example of what is called a ``separation" property. Other types of separation properties describe different types of topological spaces. These separation properties determine what kind of objects we can separate with disjoint open sets -- e.g., points, points and closed sets, closed sets and closed sets. The following are the most widely used separation properties. These properties rule out kinds of unwelcome properties that a topological space might have. For example, recall that limits of sequences are unique in Hausdorff spaces. (We traditionally call these separation properties ``axioms" because we generally assume that our topological spaces have these properties. However, these are not axioms in the usual sense of the word, but rather properties.)

\begin{definition} Let $X$ be a topological space. 
\begin{enumerate}

\item The space $X$ is a \textbf{\emph{T}$_1$-space}\index{$T_1$-space} or \textbf{Frechet space}\index{Frechet space} if for every $x\neq y$ in $X$, there exist an open set $U$ containing $y$ such that $x \notin U$.  

\item The space $X$ is a \textbf{\emph{T}$_2$-space}\index{$T_2$-space} or a \textbf{Hausdorff space} if for every $x\neq y$ in $X$, there exist disjoint open sets $U$ and $V$ such that $x\in U$ and $y\in V$.

\item The space $X$ is \textbf{regular}\index{regular topological space} if for each closed set $C$ of $X$ and each point $x \in X \setminus C$, there exists disjoint open sets $U$ and $V$ in $X$ such that $C \subseteq U$ and $x \in V$. The space $X$ is a \textbf{\emph{T}$_3$-space}\index{$T_3$-space} or a \textbf{regular Hausdorff space} if $X$ is a regular $T_1$ space. 

\item The space $X$ is a \textbf{normal}\index{normal topological space} space if for each pair $C$ and $D$ of disjoint closed subsets of $X$ there exist disjoint open sets $U$ and $V$ such that $C \subseteq U$ and $D \subseteq V$.  The space $X$ is a \textbf{\emph{T}$_4$-space}\index{$T_4$-space} or a \textbf{normal Hausdorff space} if $X$ is a normal $T_1$ space. 

\end{enumerate}
\end{definition}
The use of the variable $T$ comes from the German ``Trennungsaxiome" for separation axioms. Note again that these are not technically axioms, but rather properties. An interesting question is why we insist that $T_3$ and $T_4$-spaces also be $T_1$. We want these axioms to provide more separation at the index increases. Consider a space $X$ with the indiscrete topology. In this space, nothing is separated. However, this space is vacuously regular and normal. To avoid this seeming incongruity, we insist on working only with $T_1$ spaces. Note that a space with the indiscrete topology is not $T_1$. 

It is the case that every $T_4$-space is $T_3$, every $T_3$-space is $T_2$, and every $T_2$-space is $T_1$. Verification of these statements are left to Exercise (\ref{ex:T_1_2_3}). These properties are also all different. That is, there are $T_1$-spaces that are not $T_2$ and $T_2$-spaces that are not $T_3$. These problems are given in Exercise (\ref{ex:not_T_1_2_3}). The fact that there are $T_3$-spaces that are not $T_4$ is a bit difficult. An example is the \emph{Niemytzki plane}. The Niemytzki plane is the upper half plane $X = \{(x,y) \in \R^2 \mid y \geq 0\}$. Let $L$ be the boundary of $X$. That is, $L = \{(x,y) \in \R^2 \mid y = 0\}$.  A basis for the topology on $X$ consists of the standard open disks centered at points with $y > 0$ along with the open disks in $X \setminus L$ that are tangent to $L$ together with their points of tangency. We won't verify that the Niemytzki plane is $T_3$ but not $T_4$. The interested reader can find an accessible proof in the article ``Another Proof that the Niemytzki Plane is not Normal" by David H. Vetterlein in the \emph{Pi Mu Epsilon Journal}, Vol. 10, No. 2 (SPRING 1995), pp. 119-121. 
 
 

\csection{Summary}
Important ideas that we discussed in this section include the following.
\begin{itemize}
\item A subset $C$ of a topological space $X$ is closed if $X \setminus C$ is open.
\item Any intersection of closed sets is closed, while unions of finitely many closed sets are closed. 
\item A sequence in a topological space $X$ is a function $f: \Z^+$ to $X$.
\item A sequence $(x_n)$ in a topological space $X$ converges to a point $x$ in $X$ if for each open set $O$ containing $x$, there exists a positive integer $N$ such that $x_n \in O$ for all $n \geq N$. 
\item If a sequence $(x_n)$ in a topological space $X$ converges to a point $x$, then $x$ is a limit of the sequence $(x_n)$. 
\item A limit point of a subset $A$ of a topological space $X$ is a point $x \in X$ such that every neighborhood of $x$ contains a point in $A$ different from $x$. A subset $C$ of a topological space $X$ is closed if and only if $C$ contains all of its limit points.   
\item A boundary point of a subset $A$ of a topological space $X$ is a point $x \in X$ such that every neighborhood of $x$ contains a point in $A$ and a point in $X \setminus A$. The boundary of $A$ is the set 
\[\Bdry(A) = \{x \in X \mid x \text{ is a boundary point of } A\}.\]
A subset $C$ of $X$ is closed if and only if $C$ contains its boundary. 
\item A topological space $X$ is Hausdorff if we can separate distinct points with open sets in the space. That is, if for each pair $x$, $y$ of distinct points in $X$, there exists open sets $O_x$ of $x$ and $O_y$ of $y$ such that $O_x \cap O_y = \emptyset$. 
Hausdorff spaces are important because sequences have unique limits in Hausdorff spaces and single point sets are closed. 
\item Separation axioms tell us what kinds of objects can be separated by open sets. 
	\begin{itemize}
	\item In a $T_1$-space, we can separate two distinct points with one open set. That is, given distinct points $x$ and $y$ in a $T_1$ topological space $X$, there is an open set $U$ that separates $y$ from $x$ in the sense that $y \in U$ but $x \notin U$. 
	\item In a $T_2$-space $X$ we can separate points more distinctly. That is, if $x$ and $y$ are different points in $X$, there exist disjoint open sets $U$ and $V$ such that $x \in U$ and $y \in V$. 
	\item In a $T_3$-space $X$ we can separate a point from a closed set that does not contain that point. That is, if $C$ is a closed subset of $X$ and $x$ is a point not in $C$, there exists disjoint open sets $U$ and $V$ in $X$ such that $C \subseteq U$ and $x \in V$. 
	\item In a $T_4$-space $X$ we can separate disjoint closed sets. That is, if $C$ and $D$ are disjoint closed subsets of $X$, there exist disjoint open sets $U$ and $V$ such that $C \subseteq U$ and $D \subseteq V$.  
	\end{itemize}
	
\end{itemize}

\csection{Exercises}

\be

\item Determine exactly which finite topological spaces are Hausdorff. Prove your result.

\begin{comment}

\ExerciseSolution We will show that a finite topological space $(X,\tau)$ is Hausdorff if and only if $\tau$ is the discrete topology. Let $(X, \tau)$ be a finite topological space. Suppose that $X$ is Hausdorff. To prove that $\tau$ is the discrete metric, we show that every singleton set is open. Let $x \in X$. Since $X$ is Hausdorff, for every $y \in X$ there are open sets $O_x$ and $O_y$ such that $x \in O_x$, $y \in O_y$, and $O_x \cap O_y = \emptyset$.  Let $O = \bigcup_{\substack{y \ in X \\ y \neq x}} O_y$. Since there are only finite many points $y$, the set $O$ is a finite union of open sets and so $O$ is an open set. For each $y \in X \setminus \{x\}$, we know that $y  \in O_y$. Also, no $O_y$ contains $x$. So $O = X \setminus \{x\}$. Thus, $\{x\} = X \setminus O$ and so $\{x\}$ is an open set. It follows that $\tau$ is the discrete metric.

Now suppose that $\tau$ is the discrete metric. To show that $X$ is Hausdorff, let $x$, $y$ be distinct points in $X$. Then $\{x\}$ and $\{y\}$ are open and separate $x$ and $y$, so $X$ is Hausdorff. 

\end{comment}

\item Let $(X, \tau)$ be a topological space and let $A$ be a subset of $X$. Prove that $\overline{A} = A \cup \Bdry(A)$. 

\begin{comment}

\ExerciseSolution Let $A$ be a subset of a topological space $X$. To prove that $\overline{A} = A \cup \Bdry(A)$ we demonstrate the containment in both directions. Let $x \in \overline{A}$. Then $x \in A$ or $x \in A'$. We consider the cases.
\begin{itemize}
\item Suppose that $x \in A$. Then $x \in A \cup \Bdry{A}$ and we are done.
\item Suppose $x \notin A$ and $x \in A'$. We show that $x \in \Bdry(A)$. Let $N$ be a neighborhood of $x$. Since $x \in A'$, we know that $N$ contains a point in $A$ different than $x$. But $x \notin A$, so $N$ contains a point in $A$ and a point not in $A$. Thus, $x \in \Bdry(A) \subseteq A \cup \Bdry(A)$. 
\end{itemize}
In either case we have $x \in A \cup \Bdry(A)$ and so $\overline{A} \subseteq A \cup \Bdry(A)$. 

For the reverse containment, let $x \in A \cup \Bdry(A)$. Then $x \in A$ or $x \in \Bdry(A)$. We consider the cases.
\begin{itemize}
\item Suppose that $x \in A$. Then $x \in A \cup A'$ and we are done.
\item Suppose $x \notin A$ and $x \in \Bdry(A)$. We show that $x \in A'$. Let $N$ be a neighborhood of $x$. Since $x \in \Bdry(A)$, we know that $N$ contains a point in $A$ and a point in $X \setminus A$. But $x \notin A$, so $N$ contains a point in $A$ different from $x$. Thus, $x \in A' \subseteq \overline{A}$. 
\end{itemize}
In either case we have $x \in \overline{A}$ and so $A \cup \Bdry(A) \subseteq \overline{A}$. Combining the containments gives us $A \cup \Bdry(A) =\overline{A}$.

\end{comment}

\item Let $A$ a subset of a topological space. Prove that $\Bdry(A) = \emptyset$ if and only if $A$ is open and closed.

\begin{comment}

\ExerciseSolution Let $A$ a subset of of a topological space $X$. First we show that $\Bdry(A) = \emptyset$ implies that $A$ is open and closed. Assume $\Bdry(A) = \overline{A} \cap \overline{X \setminus A} = \emptyset.$ To prove that $A$ is closed, we will demonstrate that $A = \overline{A}.$ Since we know that $A \subset \overline{A}$ for any set $A$, we only need show that $\overline{A} \subset A.$ We proceed by contradiction. Assume $\overline{A} \not\subset A,$ that is there exists a point $x \in \overline{A} \setminus A.$ Thus, $x$ is in both $\overline{A}$ and $X \setminus A$. We know that $X \setminus A \subset \overline{X \setminus A},$ so $x \in \overline{X \setminus A}.$ Therefore, $x \in \overline{A} \cap \overline{X \setminus A} = \Bdry(A).$ This is a contradiction to our assumption that $\Bdry(A) = \emptyset.$ Therefore we must have that $\overline{A} \subset A$ and $A$ is closed. 

To show $A$ is open, we will demonstrate that $X \setminus A$ is closed. Let $B = X \setminus A$. Note that 
\[\Bdry(B) = \overline{B} \cap \overline{X \setminus B)} = \overline{X \setminus A} \cap \overline{A} = \Bdry(A) = \emptyset.\]
The proof in the previous paragraph shows that $B = X \setminus A$ is closed. Therefore, $A$ is open. 

For the converse, assume that $A$ is both open and closed. Then $A$ and $X \setminus A$ are closed. So $\overline{A} = A$ and $\overline{X \setminus A} = X \setminus A$. So 
\[\Bdry(A) = \overline{A} \cap \overline{X \setminus A} = A \cap (X \setminus A) = \emptyset.\]

\end{comment}


\item Let $X$ be a nonempty set with at least two elements and let $p$ be a fixed element in $X$. Let $\tau_p$ be the particular point topology and $\tau_{\overline{p}}$ the excluded point topology on $X$. That is
\begin{itemize}
\item $\tau_{p}$ is the collection of subsets of $X$ consisting of $\emptyset$, $X$, and all of the subsets of $X$ that contain $p$.  
\item $\tau_{\overline{p}}$ is the collection of subsets of $X$ consisting of $\emptyset$, $X$, and all of the subsets of $X$ that do not contain $p$. 
\end{itemize}
That the particular point and excluded point topologies are topologies is the subject of Exercises (\ref{ex:particular_point_topology}) and (\ref{ex:excluded_point_topology}) on page \pageref{ex:particular_point_topology}. 

Let $A = (0,1]$ be a subset of $\R$. Find, with proof, $\overline{A}$, $\Int(A)$, and $\Bdry(A)$ when
	\ba
	\item $\R$ has the topology $\tau_{p}$ with $p = 0$
	
	\item $\R$ has the topology $\tau_{\overline{p}}$ with $p = 0$.
	
	\ea
	
\begin{comment}

\ExerciseSolution

	\ba
	\item 
	\begin{itemize}
	\item Since $\R \setminus A = (-\infty,0] \cup (1,\infty)$ contains $0$, we see that $\R \setminus A$ is open. This makes $A$ a closed set, so $\overline{A} = A$. 
	\item We know that $\Int(A)$ is the largest open set contained in $A$. But no nonempty subset of $A$ is open, since $0 \notin A$. Thus, $\Int(A) = \emptyset$. 
	\item Recall that $\Bdry(A) = \overline{A} \cap \overline{\R \setminus A}$. To determine $\overline{\R \setminus A}$, we note that $\overline{\R \setminus A}$ is the smallest closed set that contains $\R \setminus A$. Any set $B$ that contains $\R \setminus A$ also contains $0$, so $\R \setminus B$ does not contain $0$ and is therefore not open. Thus, the only closed set that contains $\R \setminus A$ is $\R$, and so $\overline{\R \setminus A} = \R$. This makes $\Bdry(A) = \overline{A} \cap \overline{\R \setminus A} = \overline{A} = A$. 
	\end{itemize}
	
\item 
	\begin{itemize}
	\item To find $\overline{A}$, we use the fact that $\overline{A}$ is the smallest closed set that contains $A$. Let $B$ be a closed set that contains $A$. Since $B$ is closed, we know that $\R \setminus B$ is open and that $0$ is not in $\R \setminus B$. Thus, $0 \in B$. The smallest set that contains both $A$ and $0$ is $[0,1]$, so $\overline{A} = [0,1]$. 
	\item Since $A$ does not contain $0$, it is the case that $A$ is open and so $\Int(A) = A$. 
	\item The fact that $A$ is open means that $\R \setminus A$ is closed and so $\overline{\R \setminus A} = \R \setminus A$. Then 
	\[\Bdry(A) = \overline{A} \cap \overline{\R \setminus A} = [0,1] \cap \left((-\infty,0] \cup (1,\infty)\right) = \{0\}.\]
	\end{itemize}
	
	\ea


\end{comment}



\item \label{ex:Closed_Sets_Sorgenfrey} Let $\B =  \{[a,b) \mid a < b \text{ in } \R\}$. 
\ba
\item Show that $\B$ is a basis for a topology $\tau_{\ell \ell}$ on $\R$. This topology is called the \emph{lower limit}\index{topology!lower limit} topology on $\R$. The line $\R$ with the topology $\tau_{\ell \ell}$ is sometimes called the \emph{Sorgenfrey line}\index{Sorgenfrey line} (after the mathematician Robert Sorgenfrey).

\item Show that every open interval $(a,b)$ is also an open set in the lower limit topology.

\item If $\tau_1$ and $\tau_2$ are topologies on a set $X$ such that $\tau_1 \subseteq \tau_2$, then $\tau_1$ is said to be a \emph{coarser} topology that $\tau_2$, or $\tau_2$ is a \emph{finer} topology that $\tau_1$. Part (b) shows that the lower limit topology may be a finer topology than the Euclidean metric topology. Determine if this is true, that the lower limit topology is actually a finer topology than the Euclidean metric topology on $\R$. Justify your answer.

\item Let $a < b$ be in $\R$. Is the set $[a,b)$ clopen in $(\R, \tau_{\ell \ell})$? Prove your answer.

\ea

\begin{comment}

\ExerciseSolution

\ba
\item Any real number $a$ is in the interval $[a,a+1)$. Suppose $x \in B_1 \cap B_2$, where $B_1 = [a_1, b_1)$ and $B_2 = [a_2, b_2)$ for some real numbers $a_1$, $b_1$, $a_2$, and $b_2$. Then $a_1 \leq x < b_1$ and $a_2 \leq x < b_2$. Without loss of generality, assume that $b_1 < b_2$. Then $x \in [x,b_1) \subseteq B_1 \cap B_2$. We conclude that $\B$ is a basis for a topology on $\R$. 

\item Let $a < b$ be in $\R$. Let $N$ be in $\Z^+$ such that $a+\frac{1}{n} < b$. For $n \geq N$ in $\Z^+$, let $B_n = \left[a+\frac{1}{n}, b\right)$. Then $B_n \in \B$ by definition. We will show that $\bigcup_{n \geq N} B_n = (a,b)$. Since $B_n \subset (a,b)$ for each $n \geq N$, we see that $\bigcup_{n \geq N} B_n \subseteq (a,b)$. Now we demonstrate that $(a,b) \subseteq \bigcup_{n \geq N} B_n$. Let $x \in (a,b)$. So $a < x < b$. Let $M \in \Z^+$ such that $M \geq N$ and $a + \frac{1}{M} < x$. Then $x \in \left[a + \frac{1}{M}, b\right)$ and so $x \in \bigcup_{n \geq N} B_n$. Therefore, $(a,b) = \bigcup_{n \geq N} B_n$ and $(a,b)$ is a union of open sets. We conclude that $(a,b)$ is an open set in the lower limit topology.  

\item We will demonstrate that the set $[0,1)$, which is open in the lower limit topology, is not open in the Euclidean metric topology. This will prove that the lower limit topology is finer than the Euclidean metric topology on $\R$.  

Let $O$ be an open set in the Euclidean metric topology that contains $0$. Then $O$ contains an open ball $B(0,\epsilon)$ for some $\epsilon > 0$. But then $O$ contains $-\frac{\epsilon}{2}$, which is not in $[0,1)$. Thus, there is no open set $O$ in the Euclidean metric topology with the property that $0 \in O \subseteq [0,1)$. We conclude that $0$ is not an interior point of $[0,1)$ in the Euclidean metric topology and so $\Int([0,1)) \neq [0,1)$. So $[0,1)$ is not an open set in the Euclidean metric topology.

\item First note that if $c \in \R$, then $[c,\infty) = \bigcup_{n \in \Z^+} [c,c+n)$, so $[c,\infty)$ is an open set in the lower limit topology. Second, we have $(-\infty, c) = \bigcup_{n \in \Z^+} [c-n, c)$, so the interval $(-\infty, c)$ is also open in the lower limit topology. It follows that 
\[\R \setminus [a,b) = (-\infty, a) \cup [b,\infty)\]
is an open set in the lower limit topology. Thus, $[a,b)$ is both open and closed in $(\R, \tau_{\ell \ell})$. 


\ea

\end{comment}

\item A subset $A$ of a topological space $X$ is said to be \emph{dense} in $X$ if $\overline{A} = X$. 
	\ba
	\item Show that $\Q$ is dense in $\R$ using the Euclidean metric topology.
	
	\item Is $\Z$ dense in $\R$  using the Euclidean metric topology? Prove your answer.
	
	\item Let $A$ be a subset of a topological space $A$. Prove that $A$ is dense in $X$ if and only if $A \cap O \neq \emptyset$ for every open set $O$.
	
	\ea
	
\begin{comment}

\ExerciseSolution

	\ba
	\item Let $x$ be a real number and let $O$ be an open set containing $x$. There exists $\epsilon > 0$ such that $B(x, \epsilon) \subseteq O$. Then $y =x + \frac{\epsilon}{2} \in B(x,\epsilon)$. We know that there is a rational number between any two real numbers, so there is a rational number $r$ between $x$ and $y$. Since 
	\[d(x,y) = d(x,r) + d(r,y)\]
	we have that 
	\[d(x,r) = d(x,y) - d(r,y) < \epsilon.\]
	Thus, $r$ is a rational number in $O$ that is different from $x$. We conclude that $x \in \overline{\Q}$ and so $\overline{\Q} = \R$. 
	
	\item The subset $\Z$ is not dense in $\R$. The open set $B(0,1)$ contains no integers other than $0$, so $0 \notin \overline{\Z}$. 
	
	\item Let $A$ be a subset of a topological space $A$. Suppose $A$ is dense in $X$. Then every point in $A$ is a limit point of $A$. Let $O$ be an open set and let $x \in O$. Since $x$ is a limit point of $A$, $O$ must contain a point in $A$ different from $x$. It follows that $A \cap O \neq \emptyset$. 
	
Now assume that $A \cap O \neq \emptyset$ for every open set $O$. If $A = X$, then $\overline{A} = X$ and we are done. Suppose $A \neq X$ and let $x \in X \setminus A$. Let $O$ be an open set that contains $x$. Since $x \notin A$, the fact that $A \cap O \neq \emptyset$ implies that $O$ contains an element of $A$ different from $x$.  Thus, $x \in A'$. Therefore, $\overline{A} = A \cup A' = X$ and $A$ is dense in $X$. 
	
	\ea
	
\end{comment}

\item Let $X$ be a topological space and let $A$ be a subset of $X$. 
\ba
\item Show that the sets $\Int(A)$, $\Bdry(A)$, and $\Int(A^c)$ are mutually disjoint (that is, the intersection of any two of these sets is empty).

\item Prove that $X = \Int(A) \cup \Bdry(A) \cup \Int(A^c)$.

\ea

\begin{comment}

\ExerciseSolution 

\ba
\item We take each intersection in turn.
\begin{itemize}
\item Suppose $x \in (\Int(A) \cap \Bdry(A))$. Since $x \in \Int(A)$, there exists an open set $O$ such that $x \in O \subseteq A$. So $O \cap (X \setminus A) = \emptyset$. But $x \in \Bdry(A)$, and so $O$ must intersect $X \setminus A$. This contradiction allows us to conclude that $\Int(A) \cap \Bdry(A) = \emptyset$. 

\item Suppose $x \in \Int(A) \cap \Int(A^c)$. Since $x \in \Int(A)$, there exists an open set $O$ such that $x \in O \subseteq A$. So $O \cap (X \setminus A) = \emptyset$. Also, $x \in \Int(A^c)$ and so there is an open set $O'$ such that $x \in O' \subseteq A^c$. But this makes $x \in A$ and $x \in A^c$, a contradiction. We conclude that $\Int(A) \cap \Int(A^c) = \emptyset$. 

\item Suppose $x \in \Bdry(A) \cap \Int(A^c)$. The fact that $x \in \Int(A^c)$ means that there is an open set $O$ such that $x \in O \subseteq(A^c)$. But then $O \cap A = \emptyset$, which contradicts that fact that $x \in \Bdry(A)$. We conclude that $\Bdry(A) \cap \Int(A^c) = \emptyset$.

\end{itemize}

\item We prove that $X = \Int(A) \cup \Bdry(A) \cup \Int(A^c)$ by demonstrating the containment in both directions. Since $\Int(A)$, $\Bdry(A)$, and $\Int(A^c)$ are all subsets of $X$, we have that $\Int(A) \cup \Bdry(A) \cup \Int(A^c) \subseteq X$. For the reverse containment, let $x \in X$. If $x \in \Int(A)$ or $x \in \Int(A^c)$, we are done. So assume $x \notin \Int(A) \cup \Int(A^c)$. We will show that $x \in \Bdry(A)$. Since $x \notin \Int(A)$, no open set containing $x$ can be entirely contained in $A$. Similarly, xince $x \notin \Int(A^c)$, no open set containing $x$ can be entirely contained in $A^c$. Therefore, every open set containing $x$ must contain a point in $A$ and a point in $A^c$. Thus, $x \in \Bdry(A)$. We conclude that $X \subseteq  \Int(A) \cup \Bdry(A) \cup \Int(A^c)$. The two containments demonstrate that $X =  \Int(A) \cup \Bdry(A) \cup \Int(A^c)$. 

\ea

\end{comment}

\item \label{ex:Closed_sets:Hausdorff_subspace} Prove that a subspace of a Hausdorff space is a Hausdorff space. 

\begin{comment}

\ExerciseSolution Let $(X, \tau_X)$ be a Hausdorff topological space and let $Y$ be a subspace of $X$ with induced topology $\tau_Y$. To prove that $Y$ is a Hausdorff space, let $y_1$ and $y_2$ be distinct elements of $Y$. We will demonstrate that there exist disjoint open sets $U_1$ and $U_2$ in $Y$ such that $y_1 \in U_1$ and $y_2 \in U_2$. Now $Y \subseteq X$ and so $y_1, y_2 \in X$. Since $X$ is a Hausdorff space, there exist disjoint open sets $O_1$ and $O_2$ in $X$ with $y_1 \in O_1$ and $y_2 \in O_2$. Let $U_1 = \ O_1 \cap Y$ and $U_2 = O_2 \cap Y$. Then $U_1, U_2 \in \tau_Y$, and 
\[U_1 \cap U_2 = (O_1 \cap Y) \cap (O_2 \cap Y) = (O_1 \cap O_2) \cap Y = \emptyset.\]
The fact that $y_1 \in Y$ and $y_1 \in O_1$ implies that $y_1 \in U_1$. Similarly, $y_2 \in U_2$. Therefore, $Y$ is a Hausdorff space. 

\end{comment}

\item Let $X$ be a nonempty set with at least two elements and let $p$ be a fixed element in $X$. Let $\tau_p$ be the particular point topology and $\tau_{\overline{p}}$ the excluded point topology on $X$. That is
\begin{itemize}
\item $\tau_{p}$ is the collection of subsets of $X$ consisting of $\emptyset$, $X$, and all of the subsets of $X$ that contain $p$.  
\item $\tau_{\overline{p}}$ is the collection of subsets of $X$ consisting of $\emptyset$, $X$, and all of the subsets of $X$ that do not contain $p$.
\end{itemize}
That the particular point and excluded point topologies are topologies is the subject of Exercises (\ref{ex:particular_point_topology}) and (\ref{ex:excluded_point_topology}) on page \pageref{ex:particular_point_topology}. 

Determine, with proof, if $X$ is a Hausdorff space when 	
	\ba
	\item $X$ has the topology $\tau_{p}$
	
	\item $X$ has the topology $\tau_{\overline{p}}$.
	
	\ea
	
\begin{comment}

\ExerciseSolution

	\ba
	\item Since any two open sets in $(X, \tau_p)$ must contain $p$, it is impossible to find disjoint open sets. So $(X, \tau_p)$ is not a Hausdorff space. 
	
	\item The only open set that contains $p$ is $X$, so it is impossible to find two disjoint open sets that separate $p$ from any other element of $X$. So $(X, \tau_{\overline{p}})$ is not Hausdorff. 
	
	\ea

\end{comment}

\item \label{ex:TS_Closed_boundary} Prove that a subset $C$ of a topological space $X$ is closed if and only if $C$ contains its boundary. 


\begin{comment}

\ExxerciseSolution Let $X$ be a topological space, and let $C$ be a subset of $X$. First we assume that $C$ is closed and show that $C$ contains its boundary. Let $x \in X$ be a boundary point of $C$. We proceed by contradiction and assume that $x \notin C$. Then $x \in X \setminus C$, which is an open set. But then this neighborhood $X \setminus C$ contains no points in $C$, which contradicts the fact that $x$ is a boundary point of $C$. We conclude that $x \in C$ and $C$ contains its boundary.

For the converse, assume that $C$ contains its boundary. To show that $C$ is closed, we prove that $C$ contains its limit points. Let $x$ be a limits point of $C$. To show that $x \in C$, assume to the contrary that $x \notin C$. Then $x \in X \setminus C$, an open set. Since $X \setminus C$ is a neighborhood of each of its points, the fact that $x$ is a limit point of $C$ implies that $X \setminus C$ must contain a point of $C$, a contradiction. We conclude that $x \in C$ and $C$ contains its limit points. Therefore, $C$ is closed. 

\end{comment}

\item Recall that a point $a$ in a subset $A$ of a metric space $X$ is an isolated point of $A$ if there is a neighborhood $N$ of $a$ in $X$ such that $N \cap A = \{a\}$. We can make the same definition in any topological space.

\begin{definition} A point $a$ in a subset $A$ of a topological space $X$ is an isolated point of $A$\index{isolated point} if there is a neighborhood $N$ of $a$ such that $N \cap A = \{a\}$. 
\end{definition}

	\ba

\item If $A$ is a subset of a topological space $X$, prove that a point $a \in A$ is an isolated point of $A$ if and only if $\{a\}$ is an open set in $A$. 

\item We proved that in a metric space every boundary point of a set $A$ is either a limit point or an isolated point of $A$. (See Exercise \ref{ex:MS_boundary_limit_isolated} on page \pageref{ex:MS_boundary_limit_isolated}.) Is the same statement true in a topological space? Prove your answer. 

	\ea

\begin{comment}

\ExerciseSolution 

\ba

\item Suppose $a$ is an isolated point of $A$. Then there is a neighborhood $N$ of $a$ in $X$ such that $N \cap A = \{a\}$. Since $N$ is a neighborhood of $a$, there is an open set $O$ in $N$ that contains $a$. Then 
\[O \cap A \subseteq N \cap A = \{a\}\]
and so $\{a\}$ is an open set in $A$.

Conversely, suppose that $\{a\}$ is an open set in $A$. Then there is an open set $O$ in $X$ such that $O \cap A = \{a\}$. But $O$ is a neighborhood of $a$ in $X$, so $a$ is an isolated point of $A$.  

\item This statement is true and the proof is the same as it was for metric spaces. Let $A$ be a subset of a topological space $X$, and let $x \in X$ be a boundary point of $A$. We consider two cases: $x \notin A$ and $x \in A$. 
\begin{itemize}
\item Suppose $x \notin A$. Let $N$ be a neighborhood of $x$. Since $x$ is a boundary point of $A$ we know that $N$ contains a point in $X \setminus A$ and a point (necessarily different from $x$) in $A$. So $x$ is a limit point of $A$. 

\item Now suppose that $x \in A$. Since $x$ is a boundary point of $A$ we know that $N$ contains a point in $X \setminus A$ and a point in $A$ (which may just be $x$). If every neighborhood of $x$ contains a point in $A$ different from $x$, then $x$ is a limit point of $A$. Otherwise, there is a neighborhood $N$ of $x$ that contains no point in $A$ different from $x$. That is, $N \cap A = \{x\}$. In this case, $x$ is an isolated point of $A$. 

\end{itemize}

\ea

\end{comment}


\item For each integer $a$, let $a\Z = \{ka \mid k \in \Z\}$. That is, $a\Z$ is the set of all integer multiples of $a$.  That $\{a\Z \mid a \in \Z\}$ is a basis for a topology $\tau$ on $\Z$ is the topic of Exercise (\ref{ex:aZ_top}) on page \pageref{ex:aZ_top}. In this exercise work in the topological space $(\Z, \tau)$
\ba
\item Let $A = \mathbb{E}$, the set of even integers.
	\begin{enumerate}[i.]
	\item Find, with justification, $\Int(A)$.
	
	\item Find, with justification, $\overline{A}$.
	
	\end{enumerate}


	\item Let $B = \mathbb{N} = \{n \in \Z \mid n \geq 1\}$. That is, $\mathbb{N}$ is the set of natural numbers.  
	\begin{enumerate}[i.]
	\item Find, with justification, $\Int(B)$.
	
	\item Find, with justification, $\overline{B}$. 
	
	\end{enumerate}

\ea

\begin{comment}

\ExerciseSolution 

\ba

\item 
	\begin{enumerate}[i.]
	\item Since $2\Z \in \tau$ and $A = 2\Z$, $A$ is an open set. So $\Int(A) = A$.
	
	\item Before we proceed, we prove a lemma that will be useful.
	
\begin{lemma} Let $m$ be an integer. Any open set that contains $m$ must contain $m\Z$. 
\end{lemma}

\begin{proof} Let $m$ be an integer and let $O$ be an open set that contains $m$. We know that $O$ is a union of basis sets, so $m \in k\Z \subseteq O$ for some integer $k$. Since $m \in k\Z$, it follows that $m = k\ell$ for some integer $\ell$. But then $k$ divides $m$. Now we demonstrate that $m \Z \subseteq k\Z$. Let $n \in m\Z$. Then $n = ms$ for some integer $s$. From this we have $n = ms = k \ell s$ and $m \in k\Z$. So $m\Z \subseteq k\Z \subseteq O$.  
\end{proof}

We know that $\overline{A} = A \cup A'$, so to determine $\overline{A}$ we only have to know which odd integers are limit points of $A$. Let $m$ be an odd integer. Any open set that contains $m$ contains $m\Z$, and $m\Z$ contains the even integer $2m$. So $m$ is a limit point of $A$. We conclude that $\overline{A} = \Z$.  

	
	\end{enumerate}

\item 
	\begin{enumerate}[i.]

	\item Let $m \in \mathbb{N}$. If $O$ is an open set that contains $m$, then the lemma shows that $O$ contains $m\Z$.  But $m \Z$ contains $-m$, which is not in $B$. Thus, $B$ is not a neighborhood of any of its points and $\Int(B) = \emptyset$. 
		
	\item Let $m$ be a non-zero integer. If $O$ is an open set that contains $m$, then the lemma shows that $O$ contains $m\Z$.   But $m\Z$ contains infinitely many natural numbers and so contains elements of $B$ that are different from $m$. Thus, $m \in B'$. Notice that the open set $0\Z = \{0\}$ does not contain any points in $B$, so $0$ is not a limit point of $B$. This makes $\overline{B} = \Z \setminus \{0\}$.   
	
	\end{enumerate}
	
\ea

\end{comment}

\item Consider the Double Origin topology\index{topology!Double Origin} defined as follows. Let $X = \R^2 \cup \{0^*\}$, where $0^*$ is considered as a point that is not in $\R^2$ ($0^*$ is our double origin). As a basis for the open sets, we use the standard open balls for every point except $0$ and $0^*$. For the point $0$, we define open sets to be 
\[N(0,r) = \left\{(x,y) \in \R^2 \mid x^2+y^2 < \frac{1}{r^2}, y > 0\right\}  \cup \{0\}\]
and for $0^*$ we define open sets to be 
\[N(0^*, r) =  \left\{(x,y) \in \R^2 \mid x^2+y^2 < \frac{1}{r^2}, y < 0\right\}  \cup \{0^*\}.\]
So $N(0,r)$ is the top half of a disk of radius $\frac{1}{r}$ centered at the origin, excluding the $y$-axis but including the origin, and $N(0^*,r)$ is the bottom half of a disk of radius $\frac{1}{r}$ centered at the origin, excluding the $y$-axis and including the point $0^*$.
 
 \ba
 
 \item Show that the collection of sets described as a basis for the Double Origin topology is actually a basis for a topology.
 
 \item Is $X$ with the Double Origin topology Hausdorff? Prove your answer. 


\ea

\begin{comment}

\ExerciseSolution

\ba

\item By definition, every point in $\R^2$ is in some open ball. Let $\mathcal{B}$ be the collection of the presumed basis elements. Now suppose that $x \in X$ is in $B_1 \cap B_2$ for some $B_1, B_2 \in \mathcal{B}$. First consider the case that $x$ is neither $0$ nor $0^*$. Since $N(0,r) \cap N(0^*,s) = \emptyset$ for any $r, s > 0$, we only need the following subcases. 
\begin{itemize}
\item If $B_1$ and $B_2$ are the standard open balls, then as neighborhoods of each of their points there exist positive real numbers $r_1$, $r_2$ such that $B(x,r_1) \subseteq B_1$ and $B(x,r_2) \subseteq B_2$. Then $x \in B(x,\min\{r_1,r_2\}) \subseteq B_1 \cap B_2$. 

\item If one of the open sets is $N(0,r)$ and the other is a standard open ball $B$, the fact that $x \in N(0,r)$ implies $s = \frac{1}{r} - d(x,0)$ is positive. So if $y \in B(x,s)$, then 
\[d(y,0) \leq d(y,x) + d(x,0) < \left(\frac{1}{r} - d(x,0)\right)+ d(x,0) = \frac{1}{r}.\]
So $B(x,s) \subseteq N(0,r)$. Since $x \in B$, there is a positive number $t$ such that $B(x,t) \subseteq B$. From this it follows that $x \in B(x, \min\{s,t\}) \subseteq B \cap N(0,r)$. 

\item The case that one of the open sets is $N(0^*,r)$ and the other is a standard open ball $B$ is covered by the same argument as the previous case.

\end{itemize}

Now we consider the case where $x=0$. The basis elements that contain $x$ are only of the for $N(0,r)$. So if $x \in N(0,r_1) \cap N(),r_2)$, then $x \in N(0, \min\{r_1,r_2\}) \subseteq N(0,r_1) \cap N(0,r_2)$. The case where $x = 0^*$ uses the same argument. We conclude that $\mathcal{B}$ is a basis for a topology o $X$.  

\item The answer is yes. We can separate distinct points not equal to $0$ or $0^*$ as we do in $(\R^2, d_E)$. If $z \neq 0$ and $z \neq 0^*$, there is an $n$ such that $\frac{1}{n^2} < d_E(z,0)$. Let $r = d_E(z,0)$. Then $B(z,r-n) \cap N(0,n) = \emptyset$. We can similarly separate $z$ and $0^*$. Any two sets $N(0,n)$ and $N(0^*,n)$ are disjoint, so we can also separate $0$ and $0^*$. 

\ea

\end{comment}


\item 

\ba

\item Show that finite sets are closed in $\R^n$ with the Zariski topology.

\item Show that $\R^n$ with the Zariski topology is not Hausdorff. (Exercise \ref{ex:TS_Zariski} on page \pageref{ex:TS_Zariski} shows that a basis for the Zariski topology on $\R^n$ is the collection of sets of the form $\R^n \setminus Z(f)$, where $Z(f)$ is the set of zeros of the polynomial $f$ in $n$ variables.)

\ea

\begin{comment} 

\ExerciseSolution 

\ba

\item Let $S = \{s_1, s_2, \ldots, s_k\}$ be a finite subset of $\R^n$. For $i$ from $1$ to $k$ let $f_i \in \mathcal{P}_n$ be defined by $f(x) = x-s_i$. Then $Z(f_i) = \{s_i\}$. So $S = \bigcup_{1 \leq i \leq k} Z(f_i)$ which is a finite union of closed sets so is closed in $\R^n$ with the Zariski topology. 

\item Let $a$ and $b$ be in $\R^n$. Then there are basis elements $B_1$ and $B_2$ such that $a \in B_1$ and $b \in B_2$. Now $B_1 = \R^n \setminus Z(f)$ and $B_2 = \R^n \setminus Z(g)$ for some $f$ and $g$ in $\mathcal{P}_n$. Now $f$ and $g$ each have only finitely many zeros, so $Z(f)$ and $Z(g)$ are both finite sets. It follows that $B_1 \cap B_2$ is an infinite set. Thus, any two open sets that contain $a$ and $b$ must have a non-trivial intersection and so $\R^n$ with the Zariski topology is not Hausdorff.

\ea

\end{comment}

\item Consider the digital line topology $\tau_{dl}$ on $\Z$ with basis $\{B(n)\}$, where 
\[B(n) = \begin{cases} \{n\}	&\text{if $n$ is an odd integer}, \\ \{n-1,n,n+1\}	&\text{if $n$ is an even integer}. \end{cases}\]
%\footnote{This digital line topology has applications in digital processing -- see \emph{Introduction to Topology: Pure and Applied} by Colin Adams and Robert Franzosa , Pearson Education, Inc., 2008,  Sections 1.4 and 11.3. The set $\Z$ with the digital line topology is called the \emph{digital line}.} 
\ba
\item Let $A = \{-1,0,1\}$ of $(\Z, \tau_{dl})$.
	\begin{enumerate}[i.]
	\item Find the limit points and boundary points of $A$. Prove your conjectures. Is every limit point of $A$ a boundary point of $A$? Is every boundary point of $A$ a limit point of $A$? 


	\item Find $\overline{A}$ and write $X \setminus \overline{A}$ as a union of open sets. 

	\end{enumerate}
	
\item Now consider the subset $B = \{0\}$ of $(\Z, \tau_{dl})$. 
	\begin{enumerate}[i.]
	\item Find the limit points and boundary points of $B$. Prove your conjectures. Is every limit point of $B$ a boundary point of $B$? Is every boundary point of $B$ a limit point of $B$?

	\item Find $\overline{B}$ and write $X \setminus \overline{B}$ as a union of open sets. 

	\end{enumerate}
	
\ea

\begin{comment}

\ExerciseSolution

\ba
\item Let $A = \{-1,0,1\}$ of $(\Z, \tau_{dl})$.
	\begin{enumerate}[i.]
	\item  Any neighborhood of $0$ must contain $B(0) = \{-1,0,1\}$, and so every neighborhood of $0$ contains a point in $\{-1,0,1\}$ different from $0$. Similarly, any neighborhood of $2$ must contain $B(2) = \{1,2,3\}$, and so every neighborhood of $2$ contains a point in $\{-1,0,1\}$ different from $2$. Also, any neighborhood of $-2$ must contain $B(-2) = \{-3,-2,-1\}$, and so every neighborhood of $-2$ contains a point in $\{-1,0,1\}$ different from $-2$. Thus, $\{-2,0,2\} \subseteq \{-1,0,1\}'$. If $n$ is odd, then the open neighborhood $B(n) = \{n\}$ contains no points of $\{-1,0,1\}$ different than $n$. So no odd integer is a limit point of $\{-1,0,1\}$. If $n$ is an even integer different than $-2$, $0$, and $2$, then the open neighborhood $B(n) = \{n-1,n,n+1\}$ contains no points in $\{-1,0,1\}$. Therefore, $\{-1,0,1\}' = \{-2,0,2\}$. 

Any neighborhood of $2$ must contain $B(2) = \{1,2,3\}$, and so every neighborhood of $2$ contains a point in $\{-1,0,1\}$ and a point not in $\{-1,0,1\}$. Also, any neighborhood of $-2$ must contain $B(-2) = \{-3,-2,-1\}$, and so every neighborhood of $-2$ contain a point in $\{-1,0,1\}$ and a point not in $\{-1,0,1\}$. It follows that $\{-2,2\}~\subseteq~\Bdry(\{-1,0,1\})$. The open neighborhood $B(0) = \{-1,0,1\}$ contains no points not in $\{-1,0,1\}$, so $0$ is not a boundary point of $\{-1,0,1\}$.  The open neighborhoods $B(-1)$ and $B(1)$ contain no points not in $\{-1,0,1\}$ and if $n$ is odd, $|n| > 1$, then the open neighborhood $B(n) = \{n\}$ contains no points in $\{-1,0,1\}$. So no odd integer is a boundary point of $\{-1,0,1\}$. If $n$ is an even integer different than $-2$, $0$, and $2$, then the open neighborhood $B(n) = \{n-1,n,n+1\}$ contains no points in $\{-1,0,1\}$. Therefore, $\Bdry(\{-1,0,1\}) = \{-2,2\}$. 

Every boundary point of $A$ is a limit point of $A$, but notice that $0$ is a limit point of $\{-1,0,1\}$ but not a boundary point. 


	\item  The sets $B(n)$ are all open sets and  
\[\overline{\{-1,0,1\}} = A \cup A' = \{-1,0,1\} \cup \{-2,0,2\} = \{-2,-1,0,1,2\} = \Z \setminus \bigcup_{|n|\geq 3} B(n).\]
So 
\[X \setminus \overline{A} = \bigcup_{|n|\geq 3} B(n).\]


	\end{enumerate}
	
\item Now consider the subset $B = \{0\}$ of $(\Z, \tau_{bl})$. 
	\begin{enumerate}[i.]
	\item  No neighborhood $0$ can contain a point of $\{0\}$ other than $0$, so $0$ is not a limit point of $\{0\}$. If $n \neq 0$, then the open neighborhood $B(n)$ contains no points of $\{0\}$. Therefore, $\{0\}' = \emptyset$. 

Any neighborhood of $0$ must contain $B(0) = \{-1,0,1\}$, and so every neighborhood of $0$ contains a point in $\{0\}$ and a point not in $\{0\}$. So $0$ is a boundary point of $\{0\}$. If $n \neq 0$, then the open neighborhood $B(n)$ contains no points in $\{0\}$. Therefore, $\Bdry(\{0\}) = \{0\}$. Every limit point of $B$ is a boundary point of $B$, but $0$ is a boundary point of $\{0\}$ that is not a limit point of $\{0\}$.  

	\item  In this case
\[\overline{\{0\}} = B  \cup B' = \{0\} = \Z \setminus \bigcup_{|n|\geq 1} B(n).\]
So
\[X \setminus \overline{B} = \bigcup_{|n|\geq 1} B(n).\]


	\end{enumerate}
	
\ea


\end{comment}

\item \label{ex:T_1_2_3}
\ba

\item Prove that a topological space $X$ is $T_1$ if and only if each singleton set is closed.

\item Show that every $T_2$-space is $T_1$, that every $T_3$-space is $T_2$, and that every $T_4$-space is $T_3$. 

\ea

\begin{comment}

\ExerciseSolution

\ba

\item First we show that single point sets are closed in a $T_1$-space. Let $X$ be a $T_1$-space and let $x \in X$. If $y \in X$ then we know that there is an open set $U_y$ that contains $y$ with $x \notin U_y$. Then $O = \bigcup_{y \neq x} U_y$ is an open set in $X$ and $O = X \setminus \{x\}$. Thus, $\{x\}$ is closed. 

Now let $X$ be a topological space in which single point sets are closed. We demonstrate that $X$ is $T_1$. Let $x, y \in X$ with $x \neq y$. Let $U = X \setminus \{x\}$. Since $\{x\}$ is closed, $U$ is open. Also, $y \in U$ and $x \notin U$. Thus, $X$ is $T_1$. 

\item 
\begin{itemize}
\item First we show that every $T_2$-space is $T_1$. Suppose that $X$ is a $T_2$-space and let $x,y \in X$ with $x \neq y$. Then there exist disjoint open sets $U$ and $V$ with $x \in U$ and $y \in V$. So the open set $V$ contains $y$ but not $x$. Thus, $X$ is $T_1$.

\item Next we show that a $T_3$ space is also $T_2$. Suppose $X$ is a $T_3$-space and let $x, y \in X$ with $x \neq y$. Since $X$ is $T_1$, there is an open set $O$ such that $y \in O$ and $x \notin O$. Let $C = X \setminus O$. Then $y \in X \setminus C$ and so there exist disjoint open sets $U$ and $V$ in $X$ such that $C \subseteq U$ and $y \in V$. But $x \in C$ so $x \in U$. Thus we can separate $x$ and $y$ with disjoint open sets and $X$ is $T_2$. 

\item Finally, we demonstrate that every $T_4$-space is also $T_3$. Suppose $X$ is a $T_4$-space. Since $X$ is $T_1$, we only need to show that $X$ is regular. Let $x, y \in X$ with $x \neq y$. To show that $X$ is regular, let $C$ be a closed subset of $X$ and let $x \in X \setminus C$. The set $D = \{x\}$ is closed because $X$ is $T_4$. So there exist disjoint open sets $U$ and $V$ such that $C \subseteq U$ and $D \subseteq V$. Thus, we have disjoint open sets $U$ and $V$ with $C \subseteq U$ and $x \in V$. We conclude that $X$ is regular and therefore $T_3$. 

\end{itemize}

\end{comment}


\item \label{ex:not_T_1_2_3} In this exercise we illustrate spaces that are $T_1$ but not $T_2$ and $T_2$ but not $T_3$.
	\ba
	\item Show that $\R$ with the finite complement topology is $T_1$ but not $T_2$. 
	
	\item Define the space $\R_K$ as in Example \ref{exp:K_topology} to be the set of reals with topology $\tau$ with a basis that consists of the standard open intervals in $\R$ along with all sets of the form $(a,b) \setminus K$, where $(a,b)$ is any open interval and $K = \left\{\frac{1}{k} \mid k \in \Z^+\right\}$. Show that $\R_K$ is $T_2$ but not $T_3$. 
	
	\ea

\begin{comment}

\ExerciseSolution

\ba
	\item Consider $\R$ with the finite complement topology $\tau_{FC}$. If $x,y \in \R$ with $x \neq y$, then let $U = \R \setminus \{x\}$. Since $\R \setminus U = \{x\}$, we have that $U$ is open. Clearly $y \in U$ and $x \notin U$. So $(\R, \tau_{FC})$ is $T_1$. However, if $x, y \in \R$ with $x \neq y$ and $U$ and $V$ are open sets with $x \in U$ and $y \in V$, the fact that $\R \setminus U$ and $\R \setminus V$ are both finite implies that $U \cap V \neq \emptyset$. So $(\R, \tau_{FC})$ is not $T_2$. 

	\item Let $a$ and $b$ be distinct elements in $\R_K$. Without loss of generality, assume $a < b$. Then the open sets $(a-1,a+m)$ and $(b-m, b+1)$ separate $a$ and $b$, where $m = \frac{a+b}{2}$. So $\R_K$ is $T_2$. To demonstrate that $\R_k$ is not $T_3$, we show that $\R_K$ is not regular. Since $(-\infty, \infty) \setminus K$ is an open set, its complement $K$ is a closed set. We will prove that $0$ and $K$ cannot be separated by disjoint open sets. Let $V$ be an open set that contains $0$. Then $0$ is in some open interval $(a,b)$. But there exists $k \in \Z^+$ such that $\frac{1}{k} < b$, so $K \cap V \neq \emptyset$. Therefore, we cannot find disjoint open sets $U$ and $V$ such that $K \subseteq U$ and $0 \in V$. So $\R_K$ is not regular and $\R_K$ is not $T_3$.  

\ea

\end{comment}

\item For each of the following, answer true if the statement is always true. If the statement is only sometimes true or never true, answer false and provide a concrete example to illustrate that the statement is false. If a statement is true, explain why. 
	\ba
	\item Every limit point of a subset $A$ of a topological space $X$ is also a boundary point of $A$. 
	
	\item Every boundary point of a subset $A$ of a topological space $X$ is also a limit point of $A$. 

	\item If $X$ is a topological space and $A \subseteq X$ such that $\Int(A)=\overline{A}$, then $A$ is both open and closed.

	\item If $X$ is a topological space and $A$ and $B$ are subsets of $X$ with $\overline{A}=\overline{B}$ and $\Int(A) = \Int(B)$, then $A = B$.
	
	\item If $A$ and $B$ are subsets of a topological space $X$, then $\overline{A \cap B} = \overline{A} \cap \overline{B}$. 
	
	\item If $A$ and $B$ are subsets of a topological space $X$, then $\overline{A \cup B} = \overline{A} \cup \overline{B}$. 
	
	\ea

\begin{comment}

\ExerciseSolution

\ba

\item This statement is false. Let $X = \{a,b,c\}$ with $\tau =  \{\emptyset, \{a,b\}, X\}$. Let $A = \{a,b\}$. Every neighborhood of $a$ contains a point in $A$ different from $a$, so $a$ is a limit point of $A$. But $\{a,b\}$ is a neighborhood of $a$ that contains no points outside of $A$, so $a$ is not a boundary point of $A$.  

\item This statement is false. We know that a boundary point is either a limit point or an isolated point. Consider $X = (\R, d_E)$ and $A = (0,1) \cup \{2\}$. Then every neighborhood of $2$ contains a point in $A$ and a point not in $A$, so $2$ is a boundary point of $A$. However, $2$ is not a limit point of $A$ since $B(2,0.5)$ contains no points in $A$ other than $2$. 

	\item This statement is true. Recall that $\Int(A) \subseteq A \subseteq \overline{A}$. So if $\Int(A)=\overline{A}$, then $A = \Int(A)$ and $A$ is open, and $A = \overline{A}$ and $A$ is closed.

	\item This statement is false. Let $X = \R$ with the Euclidean metric topology. If $A = (0,1)$ and $B = [0,1]$, then $\overline{A} = [0,1] = \overline{B}$ and $\Int(A) = (0,1) = \Int(B)$, but $A \neq B$.
	
	\item This statement is false. Let $X = \{a,b,c\}$ with topology $\tau = \{\emptyset, \{a\}, \{b\}, \{a,b\}, \{a,c\}, X\}$. Let $A = \{a\}$ and $B = \{b\}$. The smallest closed set in $X$ that contains $A$ is $\{a,b\}$ while the smallest closed set in $X$ that contains $B$ is $\{b\}$. So $ \overline{A} \cap \overline{B} = \{b\}$. But $A \cap B =\emptyset$ and so $\overline{A \cap B} = \emptyset$.  
	
	\item This statement is true. Since $A \subseteq \overline{A} \subseteq \overline{A \cup B}$, we must have $\overline{A} \subseteq \overline{\overline{A \cup B}} = \overline{A \cup B}$. The same argument shows $\overline{B} \subseteq \overline{A \cup B}$. Thus, $\overline{A} \cup \overline{B} \subseteq \overline{A \cup B}$. 
	
For the reverse containment, the fact that $\overline{A}$ and $\overline{B}$ are closed sets implies that $\overline{A} \cup \overline{B}$ is a closed set. Now $A \subseteq \overline{A}$ and $B \subseteq \overline{B}$, so $A \cup B \subseteq \overline{A} \cup \overline{B}$. But $\overline{A \cup B}$ is the smallest closed set that contains $A \cup B$, so $\overline{A \cup B} \subseteq \overline{A} \cup \overline{B}$. 	
	
\ea

\end{comment}


\ee