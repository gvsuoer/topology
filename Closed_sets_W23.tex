\achapter{10}{Closed Sets in Metric Spaces}\label{sec:closed_sets}

\vspace*{-17 pt}
\framebox{
\parbox{\dimexpr\linewidth-3\fboxsep-3\fboxrule}
{\begin{fqs}
\item What are boundary points, limit points, and isolated points of a set in a metric space? How are they related and how are they different?
\item What does it mean for a set to be closed in a metric space?
\item What important properties do closed sets have in relation to unions and intersections?
\item How can we use closed sets to determine the continuity of a function?
\item How are limit points related to sequences?
\item How are boundary points related to sequences?
\item What is the boundary of a set in a metric space?
\item How are limit points and boundary points related to closed sets?
\item What is the closure of a set in a metric space?
\item How are closed sets related to sequences?
\end{fqs}}}

\vspace*{13 pt}

\csection{Introduction}
Once we have defined open sets in metric spaces, it is natural to ask if there are closed sets. Recall that closed intervals are important in calculus because every continuous function on a closed interval attains an absolute maximum and absolute minimum value on that interval. If we have closed sets in metric spaces, we might consider if there is some result that is similar to this for continuous functions on closed sets. In this section we introduce the idea of closed sets in metric spaces and discover a few of their properties. 

Every interval of the form $[a,b]$ in $\R$ is a closed set using the Euclidean metric. What distinguishes these closed intervals from the open intervals is that the open intervals do not contain either of their endpoints -- this is what makes an open interval a neighborhood of each of its points. In general, what makes open sets open is that they do not contain their boundaries. If an open set doesn't contain its boundary, then its complement, by contrast, should contain its boundary. This leads to the definition of a closed set. 

\begin{definition} \label{def:closed_metric_space} A subset $C$ of a metric space $X$ is \textbf{closed}\index{closed subset of a metric space} if its complement $X \setminus C$ is open. 
\end{definition}


We said that open sets are open because they do not contain their boundary and closed sets are closed because they do contain their boundary. However, we did not define what we mean by boundary. The point $a$ on the ``boundary" of an open interval of the form $O=(a,b)$ in $\R$ with the Euclidean metric has the property that every open ball that contains $a$ contains points in $O$ and points not in $O$. This is what makes the point $a$ lie on the boundary. We can also think of the point $a$ as being at the very limit of the set $O$. This motivates the next definition.

\begin{definition} Let $X$ be a metric space, and let $A$ be a subset of $X$. A \textbf{boundary point}\index{boundary point in a metric space} of $A$ is a point $x \in X$ such that every neighborhood of $x$ contains a point in $A$ and a point in $X \setminus A$. 
\end{definition}

For example, in $A=(0,1)$ as a subset of $(\R, d_E)$, the number 0 is a boundary point of $A$ because any open interval in $\R$ that contains $0$ contains points in $A$ and points not in $A$. Boundary points can arise in other ways. If $A = \{0,1\}$ as a subset of $(\R, d_E)$, then 0 is again a boundary point because any open interval in $\R$ that contains $0$ contains a point ($0$) in $A$ and points not in $A$. However, $0$ is the only point in $A$ that is contained in any open interval that contains $0$. In this case we call $0$ an \emph{isolated point} of $A$, and in the case of the set $A = (0,1)$ we call $0$ an \emph{accumulation point} or a  \emph{limit point} of $A$ (the use of the word ``limit" here will become clear later). 

\begin{definition} Let $X$ be a metric space, and let $A$ be a subset of $X$. 
\begin{enumerate}
\item An \textbf{accumulation point}\index{accumulation point in a metric space} or \textbf{limit point}\index{limit point in a metric space} of $A$ is a point $x \in X$ such that every neighborhood of $x$ contains a point in $A$ different from $x$. 
\item  An \textbf{isolated point}\index{isolated point in a metric space} of $A$ is a point $a \in A$ such that there exists a neighborhood $N$ of $a$ in $X$ with $N \cap A = \{a\}$.  
\end{enumerate}
\end{definition}

You might wonder about the use of the term ``limit point" and how limit points might be related to limits. As we will see later, limit points are limits of sequences, but the definition as we have given is one that will translate directly to toplogical spaces later. 

Note that every boundary point is either an accumulation point or an isolated point. The proof is left as an exercise.

\begin{pa} ~
\be
\item For each of the given sets $A$, find all boundary points, limit points, and isolated points. Then determine if the set $A$ is a closed set in the metric space $(X,d)$. Explain your reasoning.
	\ba
	\item $X = \R$, $d = d_E$, the Euclidean metric, $A = [0,0.5)$.

	\item $X = \{x \in \R \mid 0 < x \leq 1\}$, $d = d_E$, the Euclidean metric, $A = (0,0.5]$. 

	\item $X = \{a,b,c,e\}$, $d$ is the discrete metric defined by 
\[d(x,y) = \begin{cases} 0 &\text{if } x = y \\ 1 &\text{if } x \neq y, \end{cases}\]
and $A = \{a,b\}$. 

	\ea
	
\item Label each of the following statements as either true or false. If true, provide a convincing argument. If false, provide a specific counterexample.
	\ba
	\item Every limit point is a boundary point.
	
	\item Every boundary point is a limit point.

	\item Every limit point is an isolated point
	
	\item Every isolated point is a limit point.
	
	\item Every boundary point is an isolated point.
	
	\item Every isolated point is a boundary point.
	
	\ea
	
\ee

\end{pa}

\begin{comment}

\ActivitySolution

\be

\item  For each of the given sets $A$, find all boundary points, limit points, and isolated points. Then determine if the set $A$ is a closed set in the metric space $(X,d)$. Explain your reasoning.
	\ba
	\item The only points in $X$ for which every neighborhood contains points in $A$ and points not in $A$ are $0$ and $0.5$, so $0$ and $0.5$ are the boundary points of $A$. If $x \in [0,0.5]$, then every neighborhood of $x$ contains a point in $A$ different from $x$, so every $x$ with $0 \leq x \leq 0.5$ is a limit point of $A$. There are no isolated points of $A$. 
	
The set $A$ is not a closed set in $X$. Note that $\R \setminus A = (-\infty, 0) \cup [0.5, \infty)$. No matter what value $0 < \delta < 0.5$ has, the open ball $B(0.5, \delta)$ contains the point $0.5-\frac{\delta}{2}$, which is not contained in $\R \setminus A$. Since $\R \setminus A$ is not a neighborhood of $0.5$, the set $\R \setminus A$ is not open. If follows that $A$ is not closed in $\R$. 

	\item  The only point in $X$ for which every neighborhood contains points in $A$ and points not in $A$ is $0.5$, so $0.5$ is the only boundary point of $A$. If $x \in (0,0.5]$, then every neighborhood of $x$ contains a point in $A$ different from $x$, so every $x$ with $0 < x \leq 0.5$ is a limit point of $A$. There are no isolated points of $A$. 
	
The set $A$ is a closed set in $X$. In this case $X \setminus A  = (0,1]$. Note that every open ball in $X$ must be contained in $X$, so the open ball of radius $0.25$ in $X$ centered at $1$ has the form $(0.75, 1]$, which is entirely contained in $X \setminus A$. If $a$ is any point of $X setminus A$ other than $0$, then the open ball centered at $a$ of radius $\min\{a, 0.5-a\}$ is contained in $X \setminus A$. So $A \setminus A$ is a neighborhood of each of its points and $X \setminus A$ is an open set. Therefore, $A$ is a closed set in $X$. 

	\item  If $x \in X$ and $r < 1$, then $B(x,r) = \{x\}$. So $A$ has no boundary points. For the same reason, $A$ has no limit points, and every point of $A$ is an isolated point. 
		
Notice that if $x \in X \setminus = \{c,d\}$, then $B(x, 0.5) = \{a\}$. So $X \setminus A$ contains the open balls $B(c, 0.5)$ and $B(d, 0.5)$. Thus, $X \setminus A$ is a neighborhood of each of its points and $X \setminus A$ is an open set. It follows that $A$ is a closed set in $X$.  

	\ea
	
\item Label each of the following statements as either true or false. If true, provide a convincing argument. If false, provide a specific counterexample.
	\ba
	\item  This statement is false. Let $A = (0,1)$ in $(\R, d_E)$. Then $B(0.5,0.25)$ contains points in $A$ different from $0.5$, but no points not in $A$. So $0.5$ is a limit point of $A$ but not a boundary point of $A$.  
	
	\item  This statement is false. Consider $X = (\R, d_E)$ and $A = \{0\} \cup (1,2)\}$. Note that $B(0, 0.5) \cap A = \{0\}$, so $0$ is not a limit point of $A$. However, if $\epsilon > 0$, then $B(0,\epsilon)$ contain a point (namely $0$) in $A$ and points not in $A$. 
	
	\item This statement is false. Let $A = (0,1)$ in $(\R,d_E)$. If $\epsilon > 0$, the open ball $B(0, \epsilon)$ contains points in $A$ different from $0$. So $0$ is a limit point of $A$, but not an isolated point of $A$. 
	
	\item  This statement is false. Let $A = \{0,1\}$ in $(\R, d_E)$. Then $B(0, 0.5) \cap A = \{0\}$, so $0$ is an isolated point of $A$, but not a limit point of $A$.
	
	\item  This statement is false. Let $A = (0,1)$ in $(\R,d_E)$. If $\epsilon > 0$, the open ball $B(0, \epsilon)$ contains points in $A$ different from $0$ and points not in $A$. So $0$ is a boundary point of $A$, but not an isolated point of $A$. 
	
	\item  This statement is false. Let $X = \{0,1,2\}$ with the discrete metric, and let $A = \{0,1\}$. Then $B(0,0.5) \cap A = \{0\}$, so $0$ is an isolated point. But $B(0, 0.5)$ contains no points in $X \setminus A$, so $0$ is not a boundary point of $A$.  

	\ea

\ee

\end{comment}	



\csection{Closed Sets in Metric Spaces}

Recall that Definition \ref{def:closed_metric_space} defines a closed set in a metric space to be a set whose complement is open. We have seen that both $\emptyset$ and $X$ are open subsets of $X$. We now ask the same question, this time in terms of closed sets.  


\begin{activity} Let $X$ be a metric space.
\ba
\item Is $\emptyset$ closed in $X$? Explain.

\item Is $X$ closed in $X$? Explain.

\ea

\end{activity}

\begin{comment}

\ActivitySolution

\ba
\item Since $X \setminus \emptyset = X$ and $X$ is open in $X$, it follows that $\emptyset$ is closed in $X$. 

\item Since $X \setminus X = \emptyset$ and $\emptyset$ is open in $X$, it follows that $X$ is closed in $X$. 

\ea

\end{comment}


Note that a subset of a metric space can be both open and closed. We call such sets \emph{clopen} (for closed-open). When we discussed open sets, we saw that an arbitrary union of open sets is open, but that an arbitrary intersection of open sets may not be open (although a finite intersection of open sets is open). Since closed sets are complements of open sets, we should expect a similar result for closed sets.  

\begin{activity} \label{act:Closed_union} Let $X = \R$ with the Euclidean metric. Let $A_n = \left[\frac{1}{n}, 1-\frac{1}{n}\right]$ for each $n \in \Z^+$, $n \geq 2$.
\ba
\item What is $\bigcup_{n \geq 2} A_n$? A proof is not necessary.
		
\item Is $\bigcup_{n \geq 2} A_n$ closed in $\R$? Explain.

\ea

\end{activity}

\begin{comment}

\ActivitySolution

\ba
\item As we let $n$ go to infinity, $1-\frac{1}{n}$ will approach $1$ but not reach $1$, and $\frac{1}{n}$ will approach $0$ but not reach $0$. So $\bigcup_{n \geq 1} A_n = (0,1)$.
		
\item The complement of $\bigcup_{n \geq 1} A_n$ in $\R$ is $(-\infty,0] \cup [1,\infty)$, and $0$ is not an interior point. So $(-\infty,0] \cup [1,\infty)$ is not open and $\bigcup_{n \geq 1} A_n$ is not closed in $\R$.

\ea

\end{comment}

Activity \ref{act:Closed_union} shows that an arbitrary union of closed sets is not necessarily closed. However, the following theorem tells us what we can say about unions and intersections of closed sets. The results should not be surprising given the relationship between open and closed sets.

\begin{theorem} Let $X$ be a metric space.
\begin{enumerate}
\item Any intersection of closed sets in $X$ is a closed set in $X$.
\item Any finite union of closed sets in $X$ is a closed set in $X$. 
\end{enumerate}
\end{theorem}

\begin{proof} Let $X$ be a metric space. To prove part 1, assume that $\{C_{\alpha}\}$ is a collection of closed sets in $X$ for $\alpha$ in some indexing set $I$. DeMoivre's Theorem shows that 
\[X \setminus \bigcap_{\alpha \in I} C_{\alpha} = \bigcup_{\alpha \in I} X \setminus C_{\alpha}.\]
The latter is an arbitrary union of open sets and so it an open set. By definition, then, $\bigcap_{\alpha \in I} C_{\alpha}$ is a closed set. 

For part 2, assume that $C_1$, $C_2$, $\ldots$, $C_n$ are closed sets in $X$ for some $n \in \Z^+$. To show that $C = \bigcup_{k=1}^n C_k$ is a closed set, we will show that $X \setminus C$ is an open set. Now 
\[X \setminus \bigcup_{i=1}^n C_{i} = \bigcap_{i=1}^n X \setminus C_{i}\]
is a finite intersection of open sets, and so is an open set. Therefore, $\bigcup_{i=1}^n C_{i} $ is a closed set. 
\end{proof}


\csection{Continuity and Closed Sets}
Recall that we showed that a function $f$ from a metric space $(X,d_X)$ to a metric space $(Y,d_Y)$ is continuous if and only if $f^{-1}(O)$ is open for every open set $O$ in $Y$. We might conjecture that a similar result holds for closed sets. Since closed sets are complements of open sets, to make this connection we will want to know how $X \setminus f^{-1}(B)$ is related to $f^{-1}(Y \setminus B)$ for $B \subset Y$. 

\begin{activity} \label{act:CS_1} Let $f$ be a function $f$ from a metric space $(X,d_X)$ to a metric space $(Y,d_Y)$, and let $B$ be a subset of $Y$.
\ba
\item Let $x \in X \setminus f^{-1}(B)$.
	\begin{enumerate}[i.]
	\item What does this tell us about $f(x)$?

	\item What can we conclude about the relationship between $X \setminus f^{-1}(B)$ and $f^{-1}(Y \setminus B)$?
		
	\end{enumerate}
	
\item Let $x \in f^{-1}(Y \setminus B)$.
	\begin{enumerate}[i.]
	\item What does this tell us about $f(x)$?

	\item What can we conclude about the relationship between $X \setminus f^{-1}(B)$ and $f^{-1}(Y \setminus B)$?
		
	\end{enumerate}	

	\item What is the relationship between $X \setminus f^{-1}(B)$ and $f^{-1}(Y \setminus B)$?
	
\ea

\end{activity}

\begin{comment}

\ActivitySolution

\ba
\item Let $x \in X \setminus f^{-1}(B)$.
	\begin{enumerate}[i.]
	\item Then $x \in X$ but $x \notin f^{-1}(B)$. That means $x \in X$ but $f(x) \notin B$. 

	\item Since $f(x) \notin B$, $f(x) \in (Y \setminus B)$ or $x \in f^{-1}(Y \setminus B)$. Thus, $(X \setminus f^{-1}(B)) \subseteq f^{-1}(Y \setminus B)$.
		
	\end{enumerate}
	
\item Let $x \in f^{-1}(Y \setminus B)$.
	\begin{enumerate}[i.]
	\item Then $f(x) \in (Y \setminus B)$. So $f(x) \notin f^{-1}(B)$. 

	\item We conclude that $f(x) \in (X \setminus f^{-1}(B))$. Thus, $ f^{-1}(Y \setminus B) \subseteq (X \setminus f^{-1}(B))$. 
		
	\end{enumerate}	

	\item The two containments in (a) and (b) show that $X \setminus f^{-1}(B) = f^{-1}(Y \setminus B)$.
	
\ea

\end{comment}

Now we can consider the issue of continuity and closed sets.

\begin{activity} \label{act:CS_2} Let $f$ be a function from a metric space $(X,d_X)$ to a metric space $(Y,d_Y)$. 
\ba
\item Assume that $f$ is continuous and that $C$ is a closed set in $Y$. How does the result of Activity \ref{act:CS_1} tell us that $f^{-1}(C)$ is closed in $X$?

\item Now assume that $f^{-1}(C)$ is closed in $X$ whenever $C$ is closed in $Y$. How does the result of Activity \ref{act:CS_1} tell us that $f$ is a continuous function?

\ea

\end{activity}

\begin{comment}

\ActivitySolution

\ba
\item Since $C$ is closed, we know that $Y \setminus C$ is open. This means that $f^{-1}(Y \setminus C)$ is also open. Activity \ref{act:CS_1} tell us that $f^{-1}(Y \setminus B) = X \setminus f^{-1}(C)$. The fact that $X \setminus f^{-1}(C)$ is open implies that $f^{-1}(C)$ is closed. 

\item Suppose $O$ is an open set in $Y$. Then $Y \setminus O$ is a closed set. Activity \ref{act:CS_1} tells us that $f^{-1}(Y \setminus O) = X \setminus f^{-1}(O)$. That $X \setminus f^{-1}(O)$ is closed means that $f^{-1}(O)$ is open. Thus, $f$ is a continuous function. 

\ea

\end{comment}

The result of Activity \ref{act:CS_2} is summarized in the following theorem.

\begin{theorem} \label{thm:closed_sets_continuity_MS} Let $f$ be a function from a metric space $(X,d_X)$ to a metric space $(Y,d_Y)$. Then $f$ is continuous if and only if $f^{-1}(C)$ is closed in $X$ whenever $C$ is a closed set in $Y$.  
\end{theorem}

\csection{Limit Points, Boundary Points, Isolated Points, and Sequences}

Recall that a limit point of a subset $A$ of a metric space $X$ is a point $x \in X$ such that every neighborhood of $x$ contains a point in $A$ different from $x$. You might wonder about the use of the word ``limit" in the definition of limit point. The next activity should make this clear. 

\begin{activity} \label{act:CS_3} Let $X$ be a metric space, let $A$ be a subset of $X$, and let $x$ be a limit point of $A$. 
\ba
\item Let $n \in \Z^+$. Explain why $B\left(x, \frac{1}{n}\right)$ must contain a point $a_n$ in $A$ different from $x$. 

\item What is $\lim a_n$? Why?

\ea

\end{activity}

\begin{comment}

\ActivitySolution

\ba
\item The fact that $B\left(x, \frac{1}{n}\right)$ is a neighborhood of $x$ implies that $B\left(x, \frac{1}{n}\right)$ must contain a point $a_n$ in $A$ different from $x$. 

\item Given any $\epsilon > 0$, we can choose $N$ such that $\frac{1}{N} < \epsilon$. So $n \geq N$ implies that $d(a_n,x) < \frac{1}{n} < \frac{1}{N} < \epsilon$. So $\lim a_n = x$. 

\ea

\end{comment}

The result of Activity \ref{act:CS_3} is summarized in the following theorem.

\begin{theorem} \label{thm:CS_limit_pt} Let $X$ be a metric space, let $A$ be a subset of $X$, and let $x$ be a limit point of $A$. Then there is a sequence $(a_n)$ in $A$ that converges to $x$.
\end{theorem}

Of course, the constant sequence $(a)$ always converges to the point $a$, so every point in a set $A$ is the limit of a sequence. With limit points there is a non-constant sequence that converges to the point. We might ask what we can say about a point $a \in A$ if the only sequences in $A$ that converges to $a \in A$ are the eventually constant sequences $(a)$. (By an eventually constant sequence $(a_n)$, we mean that there is a positive integer $K$ such that for $k \geq K$, we have $a_k = a$ for some element $a$.) That is the subject of our next activity. 

\begin{activity} Let $(X,d)$ be a metric space, and let $A$ be a subset of $X$.
\ba
\item Let $a$ be an isolated point of $A$. Prove that the only sequences in $A$ that converge to $a$ are the eventually constant sequences $(a)$. 

\item Prove that if the only sequences in $A$ that converges to $a$ are the eventually constant sequences $(a)$, then $a$ is an isolated point of $A$.

\ea

\end{activity}

\begin{comment}

\ActivitySolution

\ba
\item Suppose $(a_n)$ is a sequence in $A$ that converges to $a$. Let $N$ be a neighborhood of $a$ in $X$ such that $A \cap N = \{a\}$. Since $N$ is a neighborhood of $a$, there is an open ball $B(a,\epsilon)$ that is a subset of $N$.  The fact that $(a_n)$ converges to $a$ means that there is a positive integer $K$ such that $k \geq K$ implies that $d(a_k,a) < \epsilon$. Then $a_k \in (A \cap N)$ and so $a_k = a$.  

\item We prove the contrapositive. Assume that $a$ is not an isolated point of $A$. We will show that there is a sequence in $A$ that converges to $a$ that is not eventually constant. For each positive integer $n$, we know that $B\left(x, \frac{1}{n}\right)$ is a neighborhood of $a$. Since $a$ is not a limit point of $A$, it follows that $B\left(x, \frac{1}{n}\right) \cap A$ contains an element $a_n$ of $A$ different from $a$.But then $(a_n)$ is a sequence in $A$ that converges to $a$ with $a_n \neq a$ for every $n$.  

\ea

\end{comment}

Boundary points are points that are, in some sense, situated ``between" a set and its complement. We will make this idea of ``between" more concrete soon. 

An argument just like the one in Activity \ref{act:CS_3} gives us the following result about boundary points. 

\begin{theorem} \label{thm:CS_2} Let $X$ be a metric space, let $A$ be a subset of $X$, and let $b$ be a boundary point of $A$. Then there are sequences $(x_n)$ in $X \setminus A$ and $(a_n)$ in $A$ that converge to $x$.
\end{theorem}

\csection{Limit Points and Closed Sets}

There is a connection between limit points and closed sets. The open set $(1,2)$ in $(\R, d_E)$ does not contain all of its limit points or any of its boundary points, while the closed set $[1,2]$ contains all of its boundary and limit points. This is an important attribute of closed sets. Recall that for a limit point $x$ of a subset $A$ of a metric space $X$, every neighborhood of $x$ contains a point in $A$ different from $x$. We can make the neighborhoods as small as we like so, in a sense, the limit points of $A$ that are not in $A$ are the points in $X$ that are arbitrarily close to the set $A$. We denote the set of limit points of $A$ as $A'$, and the limit points of a set can tell us if the set is closed.

\begin{theorem} \label{thm:closed_limitpoints} Let $C$ be a subset of a metric space $X$, and let $C'$ be the set of limit points of $C$. Then $C$ is closed if and only if $C' \subseteq C$.  
\end{theorem}

\begin{proof} Let $X$ be a metric space, and let $C$ be a subset of $X$. First we assume that $C$ is closed and show that $C$ contains all of its limit points. Let $x \in X$ be a limit point of $C$. We proceed by contradiction and assume that $x \notin C$. Then $x \in X \setminus C$, which is an open set. This implies that there is an $\epsilon > 0$ so that $B(x, \epsilon) \subseteq X \setminus C$. But then this neighborhood $B(x, \epsilon)$ contains no points in $C$, which contradicts the fact that $x$ is a limit point of $C$. We conclude that $x \in C$ and $C$ contains all of its limit points.

The converse of the result we just proved is the subject of the next activity.

\end{proof}


\begin{activity} Let $C$ be a subset of a metric space $X$, and let $C'$ be the set of limit points of $C$. In this activity we prove that $C$ is closed if $C$ contains all of its limit points. So assume $C' \subseteq C$. 
\ba
\item What do we need to do to show that $C$ is closed? 

\item If we proceed by contradiction to prove that $C$ is closed, we assume that $C$ is not closed. What does this tell us about $X \setminus C$?  

\item What does the conclusion of part (b) tells us?

\item How does the result of (c) contradict the assumption that $C$ contains all of its limit points? 

\ea

\end{activity}

\begin{comment}

\ActivitySolution

\ba
\item We need to show that $X \setminus C$ is open. 

\item To proceed by contradiction, we assume that $X \setminus C$ is not open.  

\item  Then there exists $x \in X \setminus C$ such that no neighborhood of $x$ is entirely contained in $X \setminus C$. 

\item This implies that every neighborhood of $x$ contains a point in $C$ and so $x$ is a limit point of $C$. It follows that $x \in C$, contradicting the fact that $x \in X \setminus C$. We conclude that $X \setminus C$ is open and $C$ is closed.

\ea

\end{comment}

\csection{The Closure of a Set}

We have seen that the interior of a set is the largest open subset of that set. There is a similar result for closed sets. For example, let $A = (0,1)$ in $(\R, d_E)$. The set $A$ is an open set, but if we union $A$ with its limit points, we obtain the closed set $C = [0,1]$. Moreover, The set $[0,1]$ is the smallest closed set that contains $A$. This leads to the idea of the \emph{closure} of a set. 

\begin{definition} The \textbf{closure}\index{closure of a set in a metric space} of a subset $A$ of a metric space $X$ is the set 
\[\overline{A} = A \cup A'.\]
\end{definition}

In other words, the closure of a set is the collection of the elements of the set and the limit points of the set -- those points that are on the ``edge" of the set. The importance of the closure of a set $A$ is that the closure of $A$ is the smallest closed set that contains $A$. 

\begin{theorem} \label{thm:closure_closed} Let $X$ be a metric space and $A$ a subset of $X$. The closure of $A$ is a closed set. Moreover, the closure of $A$ is the smallest closed subset of $X$ that contains $A$. 
\end{theorem}

\begin{proof} Let $X$ be a metric space and $A$ a subset of $X$. To prove that $\overline{A}$ is a closed set, we will prove that $\overline{A}$ contains its limit points. Let $x \in \overline{A}'$. To show that $x \in \overline{A}$, we proceed by contradiction and assume that $x \notin \overline{A}$. This implies that $x \notin A$ and $x \notin A'$. Since $x \notin A'$, there exists a neighborhood $N$ of $x$ that contains no points of $A$ other than $x$. But $A \subseteq \overline{A}$ and $x \notin \overline{A}$, so it follows that $N \cap A = \emptyset$. This implies that there is an open ball $B \subseteq N$ centered at $x$ so that $B \cap A = \emptyset$. The fact that $x \in \overline{A}'$ means that $B \cap \overline{A}$ contains a point $y$ in $\overline{A}$ different from $x$. Since $B \cap A = \emptyset$, we must have $y \in A'$. But this, and the  fact that $B$ is a neighborhood of $y$, means that $B$ must contain a point of $A$ different than $y$. But $B \cap A = \emptyset$, so we have reached a contradiction. We conclude that $x \in \overline{A}$ and $\overline{A}' \subseteq \overline{A}$. This shows that $\overline{A}$ is a closed set. 

The proof that $\overline{A}$ is the smallest closed subset of $X$ that contains $A$ is left for the next activity.
\end{proof}

\begin{activity} Let $(X,d)$ be a metric space, and let $A$ be a subset of $X$. 
\ba
\item What will we have to show to prove that $\overline{A}$ is the smallest closed subset of $X$ that contains $A$?

\item Suppose that $C$ is a closed subset of $X$ that contains $A$. To show that $\overline{A} \subseteq C$, why is it enough to demonstrate that $A' \subseteq C$? 

\item If $x \in A'$, what can we say about $x$? 

\item Complete the proof that $\overline{A} \subseteq C$.

\ea

\end{activity}

\begin{comment}

\ActivitySolution

\ba
\item We need to prove that any closed subset of $X$ that contains $A$ also contains $\overline{A}$.

\item Since $\overline{A} = A \cup A'$ and $A \subseteq C$, to show that $\overline{A} \subseteq C$ we only need to demonstrate that $A' \subseteq C$. 

\item Let $N$ be a neighborhood of $x$. Then $N$ contains a point of $A$ different than $x$.

\item Since $A \subseteq C$, it follows that $N$ contains a point of $C$ different than $x$. So $x$ is a limit point of $C$. The fact that $C$ is closed means that $C$ contains its limit points, so $x \in C$. Therefore, $A' \subseteq C$ and $\overline{A} \subseteq C$. 


\ea

\end{comment}

One consequence of Theorem \ref{thm:closure_closed} is the following.

\begin{corollary} A subset $C$ of a metric space $X$ is closed if and only if $C = \overline{C}$. 
\end{corollary}
 
We can also characterize closed sets as sets that contain their boundaries.

\begin{definition} The \textbf{boundary} $\Bdry(A)$ of a subset $A$ of a metric space $X$ is the set of all boundary points of $A$.
\end{definition}

\begin{theorem} \label{thm:Closed_boundary} A subset $C$ of a metric space $X$ is closed if and only if $C$ contains its boundary. 
\end{theorem}

The proof of Theorem \ref{thm:Closed_boundary} is left to Exercise (\ref{ex:closed_bounded}).

Recall that a boundary point of a subset $A$ of a metric space $X$ is a point $x \in X$ such that every neighborhood of $x$ contains a point in $A$ and a point in $X \setminus A$. The boundary points are those that are somehow ``between" a set and its complement. For example if $A = (0,1]$ in $\R$, the boundary of $A$ is the set $\{0,1\}$. We also have that $\overline{A} = [0,1]$, $\R \setminus A = (-\infty, 0] \cup (1, \infty)$, and $\overline{\R \setminus A} = (-\infty, 0] \cup [1, \infty)$. Notice that $\Bdry(A) = \overline{A}  \cap \overline{\R \setminus A}$. That this is always true is formalized in the next theorem.

\begin{theorem} \label{thm:Bd_between} Let $X$ be a metric space and $A$ a subset of $X$. Then
\[\Bdry(A) = \overline{A} \cap \overline{X \setminus A}.\]
\end{theorem}

\begin{proof} Let $X$ be a metric space and $A$ a subset of $X$. To prove $\Bdry(A) = \overline{A} \cap \overline{X \setminus A}$ we need to verify the containment in each direction. Let $x \in \Bdry(A)$ and let $N$ be a neighborhood of $x$. Then $N$ contains a point in $A$ and a point in $X \setminus A$. We consider the cases of $x \in A$ or $x \notin A$. 
\begin{itemize}
\item Suppose $x \in A$. Then $x \in \overline{A}$. Also, $x \notin X \setminus A$, so $N$ must contain a point in $X \setminus A$ different from $x$. That makes $x$ a limit point of $X \setminus A$ and so $X \in \overline{A} \cap \overline{X \setminus A}$. 
\item Suppose $x \notin A$. Then $x \in X \setminus A \subseteq \overline{X \setminus A}$. Also, $x \notin A$, so $N$ must contain a point in $A$ different from $x$. That makes $x$ a limit point of $A$ and so $X \in \overline{A} \cap \overline{X \setminus A}$. 
\end{itemize}
In either case we have $x \in \overline{A} \cap \overline{X \setminus A}$ and so $\Bdry(A) \subseteq \overline{A} \cap \overline{X \setminus A}$.

For the reverse implication, refer to the next activity.

\end{proof}


\begin{activity} Let $X$ be a metric space and $A$ a subset of $X$. In this activity we prove that 
\[\overline{A} \cap \overline{X \setminus A} \subseteq \Bdry(A).\]
Let $x \in \overline{A} \cap \overline{X \setminus A}$.
\ba
\item What must be true about $x$, given that $x$ is in the intersection of two sets?

\item Let $N$ be a neighborhood of $x$. As we did in the proof of Theorem \ref{thm:Bd_between}, we consider the cases $x \in A$ and $x \notin A$. 
	\begin{enumerate}[i.]
	\item Suppose $x \in A$. Why must $N$ contain a point in $A$ and a point not in $A$? What does this tell us about $x$?
		
	\item Suppose $x \notin A$. Why must $N$ contain a point in $A$ and a point not in $A$? What does this tell us about $x$?
		
	\item What can we conclude from parts i. and ii.?
		
	\end{enumerate}

\ea

\end{activity}


\begin{comment}

\ActivitySolution

\ba
\item Since $x \in \overline{A} \cap \overline{X \setminus A}$, it is the case that $x \in \overline{A}$ and $x \in \overline{X \setminus A}$.

\item 
	\begin{enumerate}[i.]
	\item Suppose $x \in A$. Then $N$ contains a point, namely $x$, in $A$. Now $x \notin X \setminus A$ and $x \in \overline{X \setminus A}$ means that $N$ contains a point in $X \setminus A$ different from $x$. So $N$ contains a point in $A$ and a point in $X \setminus A$, which implies $x \in \Bdry(A)$. 
	
	\item Suppose $x \notin A$. Then $N$ contains a point, namely $x$, in $X \setminus A$. Now $x \notin A$ and $x \in \overline{A}$ means that $N$ contains a point in $A$ different from $x$. So $N$ contains a point in $A$ and a point in $X \setminus A$, which implies $x \in \Bdry(A)$. 
		
	\item In either case we have $x \in \Bdry(A)$ and so $\overline{A} \cap \overline{X \setminus A} \subseteq \Bdry(A)$. The two containments show that $\Bdry(A) = \overline{A} \cap \overline{X \setminus A}$.

		
	\end{enumerate}

\ea

\end{comment}

\csection{Closed Sets and Limits of Sequences}

Suppose we consider a sequence $(a_n)$ in a subset $A$ of a metric space $X$ that converges to a point $x$. Must it be the case that $x \in A$? We consider this question in the next activity. 

\begin{activity} \label{act:closed_limitpoints} Let $A = (0,1)$ and $B = [0,1]$ in $(\R, d_E)$. For each positive integer $n$, let $a_n =\frac{1}{n}$. Note that the sequence $(a_n)$ is contained in both sets $A$ and $B$.
\ba
\item To what does the sequence $(a_n)$ converge in $\R$?   

\item Is $\lim a_n$ in $A$? 

\item Is $\lim a_n \in B$?

\item Name two significant differences between the sets $A$ and $B$ that account for the different responses in parts (b) and $(c)$? Respond using the terminology we have introduced in this section. 

\ea

\end{activity}

\begin{comment}

\ActivitySolution

\ba
\item The sequence $(a_n)$ converges to $0$.

\item Since $0 \notin A$, even though $(a_n)$ is in $A$ and converges, the sequence $(a_n)$ does not converge to a point in $A$.  

\item Since $0 \in B$, the sequence $(a_n)$ converges to a point in $B$. 

\item One reason for this behavior is that $B$ is closed and $A$ is not. That is, $B$ contains all of its limit points but $A$ does not. 

\ea

\end{comment}

The result of Activity \ref{act:closed_limitpoints} is encapsulated in the next theorem.

\begin{theorem} \label{thm:Closed_convergent} A subset $C$ of a metric space $X$ is closed if and only if whenever $(c_n)$ is a sequence in $C$ that converges to a point $c \in X$, then $c \in C$. 
\end{theorem}

\begin{proof} Let $X$ be a metric space and $C$ a subset of $X$. First assume that $C$ is closed. Let $(c_n)$ be a convergent sequence in $C$ with $c = \lim c_n$. So either $c \in C$ or $c$ is a limit point of $C$. Since $C$ contains its limit points, either case gives us $c \in C$. So $\lim c_n \in C$. 

The proof of the remaining implication is left to the next activity.

\end{proof}

\begin{activity} Let $X$ be a metric space and $C$ a subset of $X$. In this activity we will prove that if any time a sequence $(c_n)$ in $C$ converges to a point $c \in X$, the point $c$ is in $C$, then $C$ is a closed set. 
\ba
\item List three different ways that we can show that a subset of a metric space is closed. Which one might be relevant in this situation to show that the set $C$ is closed? 

\item Let $c$ be a limit point of $C$. What does that tell us? 

\item Complete the proof that $C$ is a closed set. 


\ea

\end{activity}

\begin{comment}

\ActivitySolution

\ba
\item Different ways to show that a subset of a metric space is closed are:
\begin{itemize}
\item Use the definition and show that the complement of the set is open.
\item Use Theorem \ref{thm:closed_limitpoints} and prove that the set contains its limit points.
\item Use Theorem \ref{thm:Closed_boundary} and show that the set contains its boundary. 
\item Use Theorem \ref{thm:Closed_convergent} and show that whenever a sequence in the subset converges to a point, the limit point must be in the subset.
\end{itemize}
In our case, we might use Theorem \ref{thm:closed_limitpoints} and prove that $C$ contains its limit points.

\item Since $c$ is a limit point of $C$, there is a sequence $(c_n)$ in $C$ that converges to $c$. 

\item It follows from our hypothesis that $c \in C$, so $C$ contains its limit points and is closed.

\ea

\end{comment}


\csection{Summary}
Important ideas that we discussed in this section include the following.
\begin{itemize}
\item Let $X$ be a metric space and $A$ a subset of $X$. 
	\begin{enumerate}[i.]
	\item A point $x \in X$ is a boundary point of $A$ if every neighborhood of $x$ contains a point in $A$ and a point in $X \setminus A$. 
	\item A point $x$ is a limit point of $A$ if every neighborhood of $x$ contains a point in $A$ different from $x$.
	\item A point $a \in A$ is an isolated point of $A$ if there is a neighborhood $N$ of $a$ such that $N \cap A = \{a\}$. 
	\end{enumerate}
	Boundary points and limit points don't need to be in the set $A$, whereas an isolated point of $A$ must be in $A$. In $A = (0,1) \cup \{2\}$ as a subset of $=(\R, d_E)$, $0$ is a boundary point but not an isolated point while $2$ is a boundary point but not a limit point. Also, $0.5$ is a limit point but neither a boundary or isolated point. With $A$ as subset of $\R$ with the discrete metric, every point of $A$ is an isolated point but no point in $\R$ is a boundary point or a limit point of $A$. So even though every boundary point is either a limit point or an isolated point, the three concepts are different.
\item A subset $A$ of a metric space $X$ is closed if $X \setminus A$ is an open set. 
\item Any intersection of closed sets is closed while finite unions of closed sets are closed. 
\item A function $f$ from a metric space $X$ to a metric space $Y$ is continuous if $f^{-1}(C)$ is a closed set in $X$ whenever $C$ is a closed set in $Y$. 
\item Let $X$ be a metric space, let $A$ be a subset of $X$, and let $x$ be a limit point of $A$. Then there is a sequence $(a_n)$ in $A$ that converges to $x$.
\item Let $X$ be a metric space, let $A$ be a subset of $X$, and let $x$ be a boundary point of $A$. Then there are sequences $(x_n)$ in $X \setminus A$ and $(a_n)$ in $A$ that converge to $x$.
\item The boundary of a subset $A$ of a metric space $X$ is the set of boundary points of $A$. 
\item A subset $A$ of a metric space $X$ is closed if and only if $A$ contains all of its limit points. Similarly, $A$ is closed if and only if $A$ contains all of its boundary points.  
\item The set of all limit points of a subset $A$ of a metric space $X$ is denoted by $A'$. The closure of $A$ is the set $\overline{A} = A \cup A'$. The closure of $A$ is the smallest closed set in $X$ that contains $A$. 
\item A subset $A$ of a metric space $X$ is closed if and only if $\lim a_n$ is in $A$ whenever $(a_n)$ is a convergent sequence in $A$. 
\end{itemize}

\csection{Exercises}

\be

\item Informal, but convincing, arguments suffice for this problem.

\ba  

\item Let $D = \{(x,y) \in \R^2 \mid x^2+y^2 \leq 1\}$ as a subset of $(\R^2,d_E)$. Note that $D$ is the unit disk in the plane. Determine all of the interior points, boundary points, accumulation points, and isolated points of $D$. Give reasons for your conclusions. Is $D$ an open set? Is $D$ a closed set? Explain.

\item Let $A = \Q$, the set of rational numbers, as a subset of $(\R,d_E)$. Determine all of the interior points, boundary points, accumulation points, and isolated points of $A$. Give reasons for your conclusions. Is $A$ an open set? Is $A$ a closed set? Explain. 

\item Let $A = \left\{\frac{1}{n} \ \left| \right. \in \Z^+\right\}$ as a subset of $(\R,d_E)$. Determine all of the interior points, boundary points, accumulation points, and isolated points of $A$. Give reasons for your conclusions. Is $A$ an open set? Is $A$ a closed set? Explain.


\ea

\begin{comment}

\ExerciseSolution 

Every neighborhood of a point $a$ contains an open ball centered at $a$. So it suffices to consider the neighborhoods of open balls. 
\ba

\item Let $ = \{(x,y) \mid x^2+y^2 = 1\}$, so that $S$ is the unit circle. Let $s \in S$.  Let $B$ be an open ball centered at $s$ of radius $r$ as illustrated at left in the figure below. This open ball contains points in $D$ different from $s$ and points not in $D$, so $s$ is a boundary point and an accumulation point of $D$, but not an isolated point or an interior point of $D$. 

Consider $s = (x,y)$ where $x^2+y^2 < 1$. The line through $s$ and the origin intersects $S$ in two points. Let $r$ be smaller than the distance from either of those points, and let $B = B(s,r)$ be the open ball centered at $s$ of radius $r$ as shown in the the middle figure below. Then $B(s,r)$ is entirely contained in $D$, so $s$ is an interior point of $D$, but not a boundary point, accumulation point, or isolated point of $D$. 

Finally, let $s \in \R^2 \setminus D$. The line through $s$ and the origin intersects $S$ in two points. Let $r$ be smaller than the distance from either of those points, and let $B = B(s,r)$ be the open ball centered at $s$ of radius $r$ as shown in the the figure below at right. Then $B(s,r)$ doesn't intersect $D$ at all. So $s$ is not a boundary point or accumulation point of $D$. Note, however, that $s$ is an interior point of $\R^2 \setminus D$. 

We conclude that the interior points of $D$ are the points of the form $(x,y)$ where $x^2+y^2 < 1$, the boundary points of $D$ are the points of $S$, the accumulation points of $D$ are all of the points in $D$, and $D$ has no isolated points. Since there are points in $D$ that aren't interior points, $D$ is not an open set. But every point in $\R^2 \setminus D$ is an interior point, so $\R^2 \setminus D$ is an open set. This makes $D$ a closed set. 

%\begin{figure}[h]
\begin{center}
\resizebox{!}{1.5in}{\includegraphics{Boundary_ex_circle_1}} \hspace{0.5in} \resizebox{!}{1.5in}{\includegraphics{Boundary_ex_circle_3}} \hspace{0.5in} \resizebox{!}{1.5in}{\includegraphics{Boundary_ex_circle_2}}
%\caption{Points in $\R^2$ in relation to $S$.}
%\label{fig:Symmetry_problem1}
\end{center}
%\end{figure}


\item Let $x$ be a real number and let $r > 0$ be a real number. Then $B(x,r) = (x-r, x+r)$ contains infinitely many rational numbers and infinitely many irrational numbers. So $x$ is a boundary point of $A$ and an accumulation point of $A$. That is, every real number is both a boundary point of $A$ and an accumulation point of $A$. For the same reasons, there are no interior point of $A$, nor are there any isolated points of $A$. 

Since not every point in $A$ is an interior point, $A$ is not an open set. 

The complement of $A$ is the set of irrational numbers. Let $x$ be an irrational number. Then $B(x,r)$ contains infinitely many rational numbers, so $x$ is not an interior point of $\R \setminus A$. Thus, $\R \setminus A$ is not open and so $A$ is not closed.

\item We consider cases. First, suppose that $a = 1$. Then $B(a,0.5)$ consists of all real numbers $x$ with $0.5 < x < 1.5$. So there are no elements in $A$ that are in $B(a,0.5)$ other than $a$. 

Now let $m \in \Z^+$ with $m > 1$, and suppose $a = \frac{1}{m}$ in $A$. Let $r = \frac{1}{m(m+1)}$. Then if $x \in B(a,r)$ we have $|x-a|  < r$. So 
\[-r < |x-a|  < r,\]
which implies that 
\[a-r < x < a+r.\]
Now $m+1 > m-1$, so $\frac{1}{m+1} < \frac{1}{m-1}$ and $\frac{1}{m(m+1)} < \frac{1}{m(m-1)}$. Thus,
\[a+r = \frac{1}{m} + \frac{1}{m(m+1)} < \frac{1}{m} + \frac{1}{m(m-1)} = \frac{m-1+1}{m(m-1)} = \frac{1}{m-1}.\]
Also,
\[a - r = \frac{1}{m} - \frac{1}{m(m+1)} = \frac{m+1-1}{m(m+1)} = \frac{1}{m+1}\]
so $\frac{1}{m+1} < x < \frac{1}{m(m-1)}$ and there are no elements in $A$ in $B(a,r)$ other than $a$. 

We conclude that every element of $A$ is a boundary point and an isolated point, and that no elements of $A$ are interior points or accumulation points. 

Now we consider elements $x$ that are in $\R \setminus A$.  First suppose $x=0$. Let $r > 0$. Then $B(x,r) = (x-r,x+r) - (-r,r)$ contains both positive and negative numbers. By the Archimedean property, there is a positive integer $N$ such that $N > \frac{1}{r}$ or $\frac{1}{N} < r$. So $\frac{1}{N} \in B(x,r)$. This makes $0$ a boundary point of $A$ and an accumulation point of $A$, but not an interior point. Note also that $0$ is not an interior point of $\R \setminus A$. 

If $x < 0$, then $B(x,|x|)$ is a subset of the negative real numbers and so doesn't intersect $A$ at all. So a negative real number is not a boundary, accumulation, isolated, or interior point of $A$. 

Finally, suppose $x > 0$. If $x > 1$, then $B(x,x-1)$ contains only real numbers larger than $1$. So a positive real number larger than $1$ is not a boundary, accumulation, isolated, or interior point of $A$. The last case to consider is to suppose that $0 < x < 1$. There is an integer $N$ such that $N < \frac{1}{x} < N+1$. So $\frac{1}{N+1} < x < \frac{1}{N}$. Letting $r = \min\left\{\frac{1}{N} - x, x - \frac{1}{N+1}\right\}$ we have that $B(x,r) \subset  \left(\frac{1}{N+1}, \frac{1}{N}\right)$ and so $B(x,r)$ contains no points of $A$. 

We conclude that the set of boundary points of $A$ is $A \cup \{0\}$, the set of accumulation points of $A$ is $A \cup \{0\}$, there are no interior points of $A$, and the set of isolated points of $A$ is $A$. 

Since $\Int(A) = \emptyset$, we conclude that $A$ is not an open set. Also, $0$ is not an interior point of $\R \setminus A$, so $\R \setminus A$ is not open, meaning that $A$ is not closed.   


\ea

\end{comment}


\item Let $(X,d)$ be a metric space. Let $a \in X$, and let $r > 0$. We know that the open ball $B(a,r) = \{x \in X \mid d(a,x) < r\}$ is an open set. Let 
\[B[a,r] = \{x \in X \mid d(a,x) \leq r\}.\]
Prove or disprove: $B[a,r]$ is a closed set in $X$.

\begin{comment}

\ExerciseSolution Let $(X,d)$ be a metric space. Let $a \in X$, and let $r > 0$. We will show that $B[a,r]$ is a closed set by showing that $X \setminus B[a,r]$ is an open set. Let $b \in X \setminus B[a,r]$. Then $d(a,b) > r$. Let $s = d(a,b)-r$, and let $B = B(b,s)$. Let $x \in B$. Then
\[d(a,b) \leq d(a,x) + d(x,b).\]
Since $d(x,b) < s$, we have 
\[d(a,x) \geq d(a,b) - d(x,b) > d(a,b) - s = d(a,b) - (d(a,b) - r) = r.\]
So $x \notin B[a,r]$ and $B \subseteq X \setminus B[a,r]$. This makes $X \setminus B[a,r]$ a neighborhood of each of its points and so $X \setminus B[a,r]$ is open. We conclude that $B[a,r]$ is a closed set.  

\end{comment}

\item Let $(X,d)$ be a metric space. We have seen that it is possible for a subset of $X$ to be both open and closed. There is a characterization of sets that are both open and closed in terms of their boundaries. Find and prove such a characterization. (Your statement should have the form: A subset $A$ of a metric space $X$ is both open and closed if and only if the boundary of $A$ is $\underline{\hspace{0.5in}}$.)

\begin{comment}

\ExerciseSolution We will prove the following statement:

\begin{center} A subset $A$ of a metric space $(X,d)$ is both open and closed if and only if the boundary of $A$ is empty.  \end{center}

Let $(X,d)$ be a metric space and let $A$ be a subset of $X$. First we assume that $A$ is both open and closed. Let $x$ be an element of the boundary of $A$. We know that $\Bdry(A) = \overline{A} \cap \overline{X \setminus A}$, so $x \in \overline{A}$ and $x \in \overline{X \setminus A}$. Since $A$ is closed, it follows that $A = \overline{A}$ and so $x \in A$. The fact that $A$ is open means that $X \setminus A$ is closed, and so $X \setminus A = \overline{X \setminus A}$. So $x \in X \setminus A$. Since we can't have $x \in A$ and $x \in X \setminus A$ simultaneously, we conclude that no such $x$ exists and that the boundary of $A$ is empty.

To prove that if $\Bdry(A) = \emptyset$, then $A$ is both open and closed we use the following claim.

\noindent \textbf{Claim:} For any subset $B$ of $X$, $\Bdry(B) = \Bdry(X \setminus B)$.

\noindent \textit{Proof of the Claim:}  Let $B$ be a subset of $X$. Let $b$ be a boundary point of $B$. Then every neighborhood of $b$ contains a point in $B$ and a point in $X \setminus B$. But this means that $b \in \Bdry(X \setminus B)$. So $\Bdry(B) \subseteq \Bdry(X \setminus B)$. Applying the same argument to the set $C = X \setminus B$ and $X \setminus C = B$, we see that $\Bdry(X \setminus B) \subseteq \Bdry(B)$. The two containments show that $\Bdry(B) = \Bdry(X \setminus B)$. \\

The fact that $\Bdry(A) = \emptyset$ implies that $A$ contains its boundary. So $A$ is closed. Similarly, the claim shows that $\Bdry(X \setminus A) = \emptyset$, so $X \setminus A$ is closed. This shows that $A$ is also open. 

\end{comment}

\item Let $A$ be a subset of a metric space. Let $A'$ be the set of limit points of $A$ and $A^i$ the set of isolated points of $A$. Prove the following.
\ba
\item $A \cup A^i = A \cup A'$
\item $A' \cap A^i = \emptyset$
\item $A \subseteq A' \cup A^i$
\item $x \in \overline{A}$ if and only if there is a sequence of points of $A$ which converges to $x$
\item $\overline{A}$ is the intersection of all closed sets that contain $A$
\item $\Int(A)$ is the union of all open sets contained in $A$
\item $\overline{A}$ is the disjoint union of $\Int(A)$ and $\Bdry(A)$ 
\item $\overline{X \setminus A} = X \setminus \Int(A)$
\item $\Int(X \setminus A) = X \setminus \overline{A}$
\ea

\begin{comment}

\ExerciseSolution 
\ba
\item Let $x \in A' \cup A^i$. So $x \in A'$ or $x \in A^i$. If $x \in A'$, then $x \in A \cup A'$ and we are done. If $x \in A^i$, then by definition, $x \in A \subseteq A \cup A'$ and we are also done. Thus, $A' \cup A^i \subseteq A \cup A'$. For the reverse containment, let $x \in A \cup A'$. If $x \in A'$, then $x \in A' \cup A^i$ and we are done. So suppose $x \in A \setminus A'$. Then $x \in A$ and $x \notin A'$. The fact that $x \notin A'$ means that there exists an $\epsilon > 0$ such that $B(x, \epsilon)$ contains no points in $A$ different from $x$. Thus, $B(x, \epsilon) \cap A = \{x\}$ and $x \in A^i$. In either case $x \in A' \cup A^i$ and $A \cup A' \subseteq A' \cup A^i$. The two containments demonstrate that $A \cup A' = A' \cup A^i$. So defining $\overline{A}$ as either $A \cup A'$ or $A' \cup A^i$ is appropriate. 

\item  To prove that $A' \cap A^i = \emptyset$, we proceed by contradiction and assume to the contrary that there is an element $a \in A' \cap A^i$. Since $a \in A^i$ there exists $\epsilon > 0$ so that $B(a,\epsilon) \cap A = \{a\}$. But then the ball $B(a,\epsilon)$ contains no points in $A$ other than $a$. This contradicts the fact that $a \in A'$. Thus, $A' \cap A^i = \emptyset$.

\item To prove $A \subseteq A' \cup A^i$ we begin by letting $a \in A$. If $a \in A'$, then we are done, since $a' \subseteq A' \cup A^i$. So assume that $a \notin A'$. That implies that there must be an $\epsilon > 0$ so that the ball $B(x,\epsilon)$ contains no point of $A$ different than $a$. But this makes $a$ and isolated point of $A$ and $a \in A^i$. In either case, $a \in A' \cup A^i$ and so $A \subseteq A' \cup A^i$.

\item Now we prove that $x \in \overline{A}$ if and only if there is a sequence of points of $A$ which converges to $x$. Let $x \in \overline{A}$. Then $x \in A$ or $x \in A'$. Suppose $x \in A$. Then the sequence $(x)$ in $A$ converges to $x$. Now suppose that $x$ is a limit point of $A$. Then for each $n \in \Z^+$ there exists a point $a_n \in B\left(x,\frac{1}{n}\right)$ different from $x$. We see that $\lim d(a_n,x) < \frac{1}{n}$, so the sequence $(a_n)$ converges to $x$. In either case we have found a sequence in $A$ that converges to $x$. For the converse, suppose that $x \in X$ and there is a sequence $(a_n)$ of points in $A$ that converge to $x$. If $x \in A$, then $x \in \overline{A}$ and we are done. So suppose $x \notin A$. To show that $x$ is a limit point of $A$, let $N$ be a neighborhood of $x$. Then there is an $\epsilon > 0$ so that $B(x, \epsilon) \subseteq N$. Since the sequence $(a_n)$ converges to $x$, there is an $N \in \Z^+$ such that $n \geq N$ implies $d(a_n, x) < \epsilon$. So $a_n \in B(x, \epsilon)$. Since $x \notin A$, we know that $a_n \neq x$ for all $n \in \Z^+$. It follows that $N$ contains a point in $A$ different from $x$. Thus, $x \in A'$ and $x \in \overline{A}$. In either case, $x \in \overline{A}$.

\item Let $\Gamma$ be the collection of closed subsets of $X$ that contain $A$, and let $B = \bigcap_{F \in \Gamma} F$. We know that $\overline{A}$ is closed and contains $A$, so $\overline{A} \in \Gamma$. Thus, $B \subseteq \overline{A}$. Now we verify the reverse containment. We proved that $\overline{A}$ is the smallest closed set that contains $A$, so $\overline{A}$ is a subset of every element in $\Gamma$. Therefore $\overline{A} \subseteq C$ and $\overline{A} \subseteq B$. Having verified the two containments, we can say that $\overline{A} = \bigcap_{F \in \Gamma} F$. 

\item Let $\Gamma$ be the collection of open subsets of $X$ that are contained in $A$, and let $B = \bigcup_{F \in \Gamma} F$. We know that $\Int{A}$ is open and is contained in $A$, so $\int(A) \in \Gamma$. Thus, $\Int(A) \subseteq B$. Now we verify the reverse containment. We proved that $\Int(A)$ is the largest open subset of $A$, and so every open subset of $A$ is contained in $\Int(A)$. Therefore, $B \subseteq \Int(A)$. Having verified the two containments, we can say that $\Int(A) = \bigcup_{F \in \Gamma} F$. 

\item To prove that $\overline{A}$ is the disjoint union of $\Int(A)$ and $\Bdry(A)$ we need to show two things: that $\overline{A} = \Int(A) \cup \Bdry(A)$ and that $\Int(A) \cap \Bdry(A) = \emptyset$. First we demonstrate that $\overline{A} = \Int(A) \cup \Bdry(A)$ be verifying the containments in both directions. Let $x \in \overline{A}$. Then $x \in A$ or $x \in A'$. Begin by assuming that $x \in A$. We will show that $x \in \Int(A) \cup \Bdry(A)$. If $x \in \Int(A)$, then we are done. So assume $x \notin \Int(A)$. Then there is no open ball centered at $x$ that is entirely contained in $A$. So every open ball centered at $x$ contains a point ($x$) in $A$ and a point not in $A$. Thus, $x \in \Bdry(A)$. So we have shown that if $x \in A$, then $x \in \Int(A) \cup \Bdry(A)$. To complete this part of our proof, we assume that $x \notin A$ but $x \in A'$. Since $x \in A'$, every open ball centered at $x$ contains a point in $A$. Thus, every open ball centered at $x$ contains a point in $A$ different from $x$ (since $x \notin A$) and a point ($x$) not in $A$. This makes $x$ a boundary point of $A$. So in any case, $x \in \Int(A) \cup \Bdry(A)$, and we conclude that $\overline{A} \subseteq \Int(A) \cup \Bdry(A)$.

For the reverse containment, assume that $x \in \Int(A) \cup \Bdry(A)$. We know that $\Int(A) \subseteq A \subseteq \overline{A}$, and we proved that $\Bdry(A) = \overline{A} \cap \overline{x \setminus A} \subseteq \overline{A}$. So it follows that $\Int(A) \cup \Bdry(A) \subseteq \overline{A}$. The two containments demonstrate that $\overline{A} = \Int(A) \cup \Bdry(A)$. 

Finally, we verify that $\Int(A) \cap \Bdry(A) = \emptyset$. Suppose to the contrary that there is an element $x \in \Int(A) \cap \Bdry(A)$. Then there is an open ball $B$ centered at $x$ that is entirely contained in $\Int(A) \subseteq A$. But then $B$ contains no points not in $A$ and $x$ is not in $\Bdry(A)$. This contradiction demonstrates that $\Int(A) \cap \Bdry(A) = \emptyset$.

\item To prove that $\overline{X \setminus A} = X \setminus \Int(A)$ we verify the containments in each direction. Let $x \in \overline{X \setminus A}$. Since $x \in X$, we need to show that $x \notin \Int(A)$. Let $B$ be an open ball centered at $x$. The fact that $x \in \overline{X \setminus A}$ means that $x \in X \setminus A$ or $x$ is a limit point of $X \setminus A$. If $x \in X \setminus A$, then $x \notin A$. So $x \notin \Int(A)$. Suppose that $x$ is a limit point of $X \setminus A$. Then $B$ must contain a point of $X \setminus A$ different from $x$. In other words, $B$ must contain a point not in $A$ and so $B$ cannot be a subset of $A$. Thus, $x \notin \Int(A)$. It follows that $\overline{X \setminus A} \subseteq X \setminus \Int(A)$.

For the reverse containment, let $x \in X \setminus \Int(A)$. Then $x \notin \Int(A)$. We will show that $x \in \overline{X \setminus A} = (X \setminus A) \cup (X \setminus A)'$. If $x \notin A$, then we are done. So assume $x \in A$. Since $x \notin \Int(A)$, no open ball centered at $x$ can be a subset of $A$. In other words, every open ball centered at $x$ must contain a point not in $A$. Thus every open ball centered at $x$ must contain a point of $X \setminus A$ that is different than $x$. This means that $x$ is a limit point of $X \setminus A$ and so $x \in \overline{X \setminus A}$. We conclude that $X \setminus \Int(A) \subseteq \overline{X \setminus A}$. The two containments verify that $\overline{X \setminus A} = X \setminus \Int(A)$.

\item To prove that $\Int(X \setminus A) = X \setminus \overline{A}$ we verify the containments in each direction. Let $x \in \Int(X \setminus A)$. Since $x \in X$, to show that $x \in X \setminus \overline{A}$ we only need demonstrate that $x \notin \overline{A}$. The fact that $x \in \Int(X \setminus A)$ implies that there is an open ball $B$ centered at $x$ that is a subset of $X \setminus A$. So $x \notin A$. Also, that this ball $B$ does not contain a point of $ A$ means that $x$ is not a limit point of $A$. So $x \notin \overline{A}$ and $x \in X \setminus \overline{A}$. Thus, $\Int(X \setminus A) \subseteq X \setminus \overline{A}$. 

For the reverse containment, let $x \in X \setminus \overline{A}$. To show that $x \in \Int(X \setminus A)$ we will show that there is an open ball centered at $x$ that is a subset of $X \setminus A$. So $x \notin A$. Now $x \notin \overline{A}$, so $x \notin A'$. This means that there is an open ball $B$ centered at $x$ that contains no points of $A$ different from $x$. Since $x \notin A$, the open ball $B$ contains no points of $A$. Thus, $B \subseteq X \setminus A$ and $x \in \Int(X \setminus A)$. Thus, $X \setminus \overline{A} \subseteq \Int(X \setminus A)$ and the two containments show that $\Int(X \setminus A) = X \setminus \overline{A}$.

\ea

\end{comment}

\item Let $(X,d)$ be a metric space and $A$ a subset of $X$. Prove that a point $x \in X$ is a limit point of $A$ if and only if every open ball centered at $x$ contains a point in $A$ different from $x$.

\begin{comment}

\ExerciseSolution Suppose that $x$ is a limit point of $A$. Let $r>0$. Since $B(x,r)$ is a neighborhood of $x$, the definition of limit point tells us that $B(x,r)$ contains a point in $A$ different from $x$. So every open ball centered at $x$ contains a point in $A$ different from $x$.

Now assume that every open ball centered at $x$ contains a point in $A$ different from $x$. Let $N$ be a neighborhood of $x$. Then $B(x,r) \subseteq N$ for some $r > 0$. By hypothesis, this open ball contains a point in $A$ different from $x$, and that point is also in $N$. Thus, every neighborhood of $x$ contains a point in $A$ different from $x$ and $x$ is a limit point of $A$.  


\end{comment}

\item Let $A$ be a subset of a metric space. Let $A'$ be the set of limit points of $A$ and $A^i$ the set of isolated points of $A$. 
\ba
\item Prove that $A' \cap A^i = \emptyset$ and $A \subseteq A' \cup A^i$. 
\item Prove that $x \in \overline{A}$ if and only if there is a sequence of points of $A$ which converges to $x$. 
\item Prove that if $F$ is a closed set such that $A \subseteq F$, then $\overline{A} \subseteq F$. Then prove that $\overline{A}$ is the intersection of all such closed sets $F$ and hence is closed.
\ea

\begin{comment}

\ExerciseSolution Before we prove these items, we demonstrate that $A \cup A^i = A \cup A'$. Let $x \in A' \cup A^i$. So $x \in A'$ or $x \in A^i$. If $x \in A'$, then $x \in A \cup A'$ and we are done. If $x \in A^i$, then by definition, $x \in A \subseteq A \cup A'$ and we are also done. Thus, $A' \cup A^i \subseteq A \cup A'$. For the reverse containment, let $x \in A \cup A'$. If $x \in A'$, then $x \in A' \cup A^i$ and we are done. So suppose $x \in A \setminus A'$. Then $x \in A$ and $x \notin A'$. The fact that $x \notin A'$ means that there exists an $\epsilon > 0$ such that $B(x, \epsilon)$ contains no points in $A$ different from $x$. Thus, $B(x, \epsilon) \cap A = \{x\}$ and $x \in A^i$. In either case $x \in A' \cup A^i$ and $A \cup A' \subseteq A' \cup A^i$. The two containments demonstrate that $A \cup A' = A' \cup A^i$. So defining $\overline{A}$ as either $A \cup A'$ or $A' \cup A^i$ is appropriate. 

\ba
\item Assume to the contrary that there is an element $a \in A' \cap A^i$. Since $a \in A^i$ there exists $\epsilon > 0$ so that $B(a,\epsilon) \cap A = \{a\}$. But then the ball $B(a,\epsilon)$ contains no points in $A$ other than $a$. This contradicts the fact that $a \in A'$. Thus, $A' \cap A^i = \emptyset$.

\item Let $a \in A$. If $a \in A'$, then we are done, since $a' \subseteq A' \cup A^i$. So assume that $a \notin A'$. That implies that there must be an $\epsilon > 0$ so that the ball $B(x,\epsilon)$ contains no point of $A$ different than $a$. Bu this makes $a$ and isolated point of $A$ and $a \in A^i$. In either case, $a \in A' \cup A^i$ and so $A \subseteq A' \cup A^i$.

\item Let $x \in \overline{A}$. Then $x \in A$ or $x \in A'$. Suppose $x \in A$. Then the sequence $(x)$ in $A$ converges to $x$. Now suppose that $x$ is a limit point of $A$. Then for each $n \in \Z^+$ there exists a point $a_n \in B\left(x,\frac{1}{n}\right)$ different from $x$. We see that $\lim d(a_n,x) < \frac{1}{n}$, so the sequence $(a_n)$ converges to $x$. In either case we have found a sequence in $A$ that converges to $x$. For the converse, suppose that $x \in X$ and there is a sequence $(a_n)$ of points in $A$ that converge to $x$. If $x \in A$, then $x \in \overline{A}$ and we are done. So suppose $x \notin A$. To show that $x$ is a limit point of $A$, let $N$ be a neighborhood of $x$. Then there is an $\epsilon > 0$ so that $B(x, \epsilon) \subseteq N$. Since the sequence $(a_n)$ converges to $x$, there is an $N \in \Z^+$ such that $n \geq N$ implies $d(a_n, x) < \epsilon$. So $a_n \in B(x, \epsilon)$. Since $x \notin A$, we know that $a_n \neq x$ for all $n \in \Z^+$. It follows that $N$ contains a point in $A$ different from $x$. Thus, $x \in A'$ and $x \in \overline{A}$. In either case, $x \in \overline{A}$.
 
We proved in class that if $F$ is a closed set such that $A \subseteq F$, then $\overline{A} \subseteq F$. So the only item left to prove is that $\overline{A}$ is the intersection of all such closed sets $F$ and hence is closed. We proved in class that $\overline{A}$ is closed, so we only need to demonstrate that $\overline{A}$ is the intersection of all closed sets that contain $A$. 

Let $\Gamma$ be the collection of closed subsets of $X$ that contain $A$, and let $B = \bigcap_{F \in \Gamma} F$. Since every set in $\Gamma$ contains $A$, then $A \subseteq B$. Now we verify the reverse containment. Suppose $b \in B$. Proceed by contradiction. Assume $b \notin \overline{A}$. Then $b$ is not in $A$ and $b$ is not a limit point of $A$. So there exists an $\epsilon > 0$ so that $B(b,\epsilon) \cap A = \emptyset$. Let $F = X \setminus B(b,\epsilon)$. We know that $B(b, \epsilon)$ is an open set, so $F$ is closed. Since $B(b,\epsilon) \cap A = \emptyset$, we have $A \subset F$. Therefore, $F$ is a closed set containing $A$. Hence, $b \in F$. But $b \notin F = X \setminus B(b,\epsilon)$, which is a contradiction. Therefore, we conclude that $b \in \overline{A}$ and $B \subset \overline{A}$. Having verified the two containments, we can say that $\overline{A} = \bigcap_{F \in \Gamma} F$. 

\ea

\end{comment}

\item Recall that the distance from a point $x$ in a metric space $X$ to a nonempty subset $A$ of $X$ is 
\[d(a,X) = \inf\{d(x,a) \mid a \in A\}.\]
Prove that a subset $C$ of a metric space $X$ is closed if and only if whenever $x \in X$ and $d(x,C) = 0$, then $x \in C$. 

\begin{comment}

\ExerciseSolution Let $C$ be a subset of a metric space $X$. First assume that $C$ is closed. We will show that whenever $x \in X$ and $d(x,C) = 0$, then $x \in C$. So let $x \in X$ such that $d(x,C) = 0$. We know that there exists a sequence $(c_n)$ in $C$ such that $0 = d(x,C) = \lim d(x,c_n)$. If $x = c_n$ for some $n$, then $x \in C$ and we are done. Otherwise, given any open ball $B$ around $x$, there is a positive integer $n$ so that $c_n \in B$. Thus, every neighborhood of $x$ contains a point in $C$ different from $x$. This makes $x$ a limit point of $C$. The fact that $C$ is closed implies $x \in C$. 

Now assume that whenever $x \in X$ and $d(x,C) = 0$, then $x \in C$. We will prove that $C$ is closed by demonstrating that $C$ contains all of its limit points. Let $x$ be a limit point of $C$. So every neighborhood of $x$ contains a point in $C$ different from $x$. Thus, there is a point $c_n \in C$ such that $c_n \in B\left(x, \frac{1}{n}\right)$ for any positive integer $n$. It follows that $d(x,c_n) < \frac{1}{n}$ and so $\lim d(x,c_n) = 0$. Thus, $d(x,C) = 0$ and so $x \in C$. We conclude that $C$ contains all of its limit points and it therefore a closed set.

\end{comment}


\item Let $(X,d)$ be a metric space. In this exercise we show that some subsets of $X$, other than $\emptyset$ and $X$ must be closed. Show that any finite subset of $X$ is closed. (Hint: What are the limits points of a finite subset?)

\begin{comment}

\ExerciseSolution Let $(X,d)$ be a metric space and let $A$ be a finite subset of $X$. Let $a \in A$. Since $A$ is finite, label the other elements of $A$ as $a_1$, $a_2$, $\ldots$, $a_n$. Let $\epsilon = \min\{d(a,a_k) \mid 1 \leq k \leq n\}$. The fact that $A$ is finite means that $\epsilon$ is positive. Since $d(a,a_k) \geq \epsilon$ for each $k$ between $1$ and $n$, it follows that $B(a,\epsilon) =\{a\}$. This means that $a$ cannot be a limit point of $A$, since this neighborhood of $a$ contains no points in $A$ different than $a$. Thus, $A$ has no limit points and so contains all of its limit points. We conclude that $A$ is closed. 

\end{comment}

\item \label{ex:closed_bounded} Prove that a subset $C$ of a metric space $X$ is closed if and only if $C$ contains its boundary. 

\begin{comment}

\ExerciseSolution Let $X$ be a metric space, and let $C$ be a subset of $X$. First we assume that $C$ is closed and show that $C$ contains its boundary. Let $x \in X$ be a boundary point of $C$. We proceed by contradiction and assume that $x \notin C$. Then $x \in X \setminus C$, which is an open set. This implies that there is an $\epsilon > 0$ so that $B(x, \epsilon) \subseteq X \setminus C$. But then this neighborhood $B(x, \epsilon)$ contains no points in $C$, which contradicts the fact that $x$ is a boundary point of $C$. We conclude that $x \in C$ and $C$ contains its boundary.

For the converse, assume that $C$ contains its boundary. To show that $C$ is closed, we prove that $X \setminus C$ is open. We again proceed by contradiction and assume that $X \setminus C$ is not open. Then there exists $x \in X \setminus C$ such that no neighborhood of $x$ is entirely contained in $X \setminus C$. This implies that every neighborhood of $x$ contains a point in $C$. Since $x \in X \setminus C$, we have that $x \notin C$. So every neighborhood of $x$ contains a point in $C$ and a point in $X \setminus C$. Thus, $x$ is a boundary point of $C$. It follows that $x \in C$, contradicting the fact that $x \in X \setminus C$. We conclude that $X \setminus C$ is open and $C$ is closed.

\end{comment}

\item Let $(X,d_X)$ and $(Y,d_Y)$ be metric space and let $f: X \to Y$ be a function. 
\ba
\item Prove that $f$ is continuous if and only if $ f^{-1}(\Int(B)) \subseteq \Int(f^{-1}(B))$ for any subset $B$ of $Y$. 

\item Give an example where the containment, and not the equality, in (a) is the best we can do.  

\item Give an example to show that the equality in (a) can actually be achieved. 

\ea

\begin{comment}

\ExerciseSolution

\ba
\item  First assume that $f$ is a continuous function. Let $B$ be a subset of $Y$. Now $\Int(B)$ is an open set, so $f^{-1}(\Int(B))$ is an open set in $X$. Now we demonstrate that $f^{-1}(\Int(B)) \subseteq f^{-1}(B)$. Let $x \in f^{-1}(\Int(B))$. Since $f(x) \in \Int(B) \subseteq B$ it follows that $x \in f^{-1}(B)$. Thus, $f^{-1}(\Int(B)) \subseteq f^{-1}(B)$. The fact that $\Int(f^{-1}(B))$ is the largest open subset of $f^{-1}(B)$ means that $f^{-1}(\Int(B)) \subseteq \Int(f^{-1}(B))$. 

Now we assume that $f^{-1}(\Int(B)) \subseteq \Int(f^{-1}(B))$ for any subset $B$ of $Y$ and prove that $f$ is a continuous function. Let $O$ be an open set in $Y$. Then $O = \Int(O)$. It follows that 
\[f^{-1}(O) = f^{-1}(\Int(O)) \subseteq  \Int(f^{-1}(O)).\]
Since the interior of any set $A$ is a subset of $A$, we also have $\Int(f^{-1}(O)) \subseteq f^{-1}(O)$. Therefore, $f^{-1}(O) = \Int(f^{-1}(O))$ and $f^{-1}(O)$ is an open set. We conclude that $f$ is a continuous function.

\item  Let $(X,d_X) = (\R, d)$ with the discrete metric and $(Y, d_Y) = (\R, d_E)$. The every function from $X$ to $Y$ is continuous. Let $f: X \to Y$ be defined by $f(x) = x$. Let $B = \{0,1\}$ in $Y$. Then $f^{-1}(B) = \{0,1\}$ is an open set in $X$ and so $\Int(f^{-1}(B)) = \{0,1\}$. However, $\Int(B) = \emptyset$, so $f^{-1}(\Int(B)) = \emptyset$ as well. So in this case $f^{-1}(\Int(B))$ is not equal to $\Int(f^{-1}(B))$ even though $f$ is a continuous function.   

\item Let $X = Y = \R$, both with the Euclidean metric, and let $F$ be the identity mapping. Let $B$ be the open interval $(0,1)$. Then $\Int(B) = B$ and $f^{-1}(\Int(B)) = \Int(B) = B = f^{-1}(B) = f^{-1}(\Int(B))$. 

\ea


\end{comment}

\item \label{ex:MS_boundary_limit_isolated} Let $(X,d)$ be a metric space and let $A$ be a subset of $X$. Prove that every boundary point of $A$ is either a limit point or an isolated point of $A$. 

\begin{comment}

\ExerciseSolution Let $A$ be a subset of a metric space $X$, and let $x \in X$ be a boundary point of $A$. We consider two cases: $x \notin A$ and $x \in A$. 
\begin{itemize}
\item Suppose $x \notin A$. Let $N$ be a neighborhood of $x$. Since $x$ is a boundary point of $A$ we know that $N$ contains a point in $X \setminus A$ and a point (necessarily different from $x$) in $A$. So $x$ is a limit point of $A$. 

\item Now suppose that $x \in A$. Since $x$ is a boundary point of $A$ we know that $N$ contains a point in $X \setminus A$ and a point in $A$ (which may just be $x$). If every neighborhood of $x$ contains a point in $A$ different from $x$, then $x$ is a limit point of $A$. Otherwise, there is a neighborhood $N$ of $x$ that contains no point in $A$ different from $x$. That is, $N \cap A = \{x\}$. In this case, $x$ is an isolated point of $A$. 

\end{itemize}

\end{comment}

\item Let $(X,d)$ be a metric space, and let $A$ and $B$ be subsets of $X$.
	\ba
	\item Is it the case that $\overline{A \cup B} = \overline{A} \cup \overline{B}$? If true, prove it. If false, show why  and prove any containment that is true. 
	
	\item Is it the case that $\overline{A \cap B} = \overline{A} \cap \overline{B}$? If true, prove it. If false, show why  and prove any containment that is true. 
	
	\ea

\begin{comment}

\ExerciseSolution Let $(X,d)$ be a metric space and let $A$ and $B$ be subsets of $X$. 
\ba
\item We will show that $\overline{A \cup B} = \overline{A} \cup \overline{B}$. We first prove that $(A \cup B)' = A' \cup B'$. Let $x \in (A \cup B)'$. Then there is a sequence $(x_n)$ in $A \cup B$ that converges to $x$. Since there are infinitely many $x_n$, at least one of the sets $A$ or $B$ much contain infinitely many of the $x_n$. Without loss of generality, assume that $A$ contains infinitely many $x_n$, and let $(a_k)$ be the sequence such that $a_k$ is the $k$th entry from the sequence $(x_n)$ that is in $A$.  We prove that $(a_k)$ converges to $x$, which shows that $x \in A'$. Let $\epsilon > 0$. Since $(x_n)$ converges to $x$, there is a positive integer $N$ such that $d(x_n,x) < \epsilon$ for $n \geq N$. But then 
\[d(a_k,x) = d(x_k,x) < \epsilon\]
for $k \geq N$. Thus $(a_k)$ converges to $x$. We conclude that $(A \cup B)' \subseteq (A' \cup B')$. 

Now suppose that $x \in A' \cup B'$. Then $x \in A'$ or $x \in B'$. Without loss of generality, assume $x \in A'$. Then there is a sequence $(a_n)$ in $A$ that converges to $x$. But $a_n \in A \subseteq A\cup B$, so the sequence $(a_n)$ is also in $A \cup B$. Thus, $x \in (A \cup B)'$ and $(A' \cup B') \subseteq (A \cup B)'$. The two containments show that $(A' \cup B') = (A \cup B)'$.  

From this result we can show that $\overline{A \cup B} = \overline{A} \cup \overline{B}$. Using the fact that $\overline{C} = C \cup C'$ for any subset of $X$, we have
\[\overline{A \cup B} = (A \cup B) \cup (A \cup B)' = (A \cup B) \cup (A' \cup B') = (A \cup A') \cup (B \cup B') = \overline{A} \cup \overline{B}.\]

\item It is not true that $\overline{A \cap B} = \overline{A} \cap \overline{B}$. For example, let $(X, d) = (\R, d_E)$ and let $A = (0,1)$ and $B = (1,2)$. Then $A \cap B = \emptyset$ and so $\overline{A \cap B} = \emptyset$. But $\overline{A} = [0,1]$ and $\overline{B} = [1,2]$, which makes $\overline{A} \cap \overline{B} = \{1\}$. 

However, it is true that $\overline{A \cap B} \subseteq \overline{A} \cap \overline{B}$. To prove this containment, let $x \in \overline{A \cap B}$. Then there is a sequence $(x_n)$ in $A \cap B$ that converges to $x$. But $x_n \in A$ and $x_n \in B$ for each $n$, so there is a sequence $(x_n)$ in $A$ that converges to $x$ and a sequence $(x_n)$ in $B$ that converges to $x$. It follows that $x \in \overline{A}$ and $x \in \overline{B}$. So $x \in \overline{A} \cap \overline{B}$. We conclude that $\overline{A \cap B} \subseteq \overline{A} \cap \overline{B}$. 
\ea

\end{comment}



\item Recall that an infinite union of closed sets in a metric space may not be closed, and that an infinite intersection of open sets in a metric space may not be open. In this exercise we explore situations in which we can conclude that an infinite union of closed sets is closed and an infinite intersection of open sets is open. Let $(X,d)$ be a metric space. 
	\ba
	\item We first establish a preliminary result. Let $C$ be a closed subset of $X$ and $x \in X$. Prove that if $x \notin C$, then $d(x,C) > 0$. 

	\item Let $\{C_{\alpha}\}$ be a collection of closed subsets of $X$ for $\alpha$ in some indexing set $I$ with the property that given any $x \in X$, there exists an $\epsilon_x > 0$ such that $B(x, \epsilon_x)$ intersects at most finitely many of the sets $C_{\alpha}$. Prove that $\bigcup_{\alpha \in I} C_{\alpha}$ is closed. 	

	\item Determine and prove an analogous statement for open sets in $X$. 
	
	\ea

\begin{comment}

\ExerciseSolution

	\ba
	\item  We proceed by contradiction and assume that $x \notin C$ and $d(x,C) = 0$. Since $C$ is closed, $X \setminus C$ is open. So there is an $\epsilon > 0$ such that $B(x,\epsilon) \subseteq X \setminus C$. So $B(x,\epsilon) \cap C = \emptyset$. This means that there are no points in $C$ within $\epsilon$ of $x$, and so $d(x,C) \geq \epsilon > 0$. 

	\item  Let $(X,d)$ be a metric space. Let $\{C_{\alpha}\}$ be a collection of closed subsets of $X$ for $\alpha$ in some indexing set $I$ with the property that given any $x \in X$, there exists an $\epsilon_x > 0$ such that $B(x, \epsilon_x)$ intersects at most finitely many of the sets $C_{\alpha}$. Let $C = \bigcup_{\alpha \in I} C_{\alpha}$. We will prove that $C$ is closed by showing that $X \setminus C$ is open. Let $x \in X \setminus C$. Then the open ball $B(x, \epsilon_x)$ intersects at most finitely many sets $C_1$, $C_2$, $\ldots$, $C_n$ in $\{C_{\alpha}\}$. Since $x \notin C$, it follows that $x \notin C_i$ for any $i$. Now each of the sets $C_i$ is closed, so $d(x,C_i) = \delta_i > 0$ for each $i$. Let $\delta = \min\{\delta_i \mid 1 \leq i \leq n\}$. Then $B(x, \delta) \cap C = \emptyset$ and so $B(x, \delta) \subseteq X \setminus C$. Thus, $x$ is an interior point of $X \setminus C$ and so $X \setminus C$ is a neighborhood of each of its points. Therefore, $X \setminus C$ is open and so $C$ is closed. 

\noindent \textbf{Alternate Proof}. To prove that $C$ is closed, we demonstrate that $C$ contains all of its limit points. Let $x$ be a limit point of $C$. By hypothesis, the open ball $B(x, \epsilon_x)$ intersects at most finitely many sets $C_1$, $C_2$, $\ldots$, $C_k$ in $\{C_{\alpha}\}$. Since these sets $C_i$ are all closed, any finite union of these sets is closed. So the set $F = \bigcup_{i=1}^k C_i$ is a closed set. Theorem 16 on our theorem sheet tells us that there exists a sequence $(c_n)$ in $C$ so that $\lim c_n = x$. Thus, there is a positive integer $N$ with the property that $d(c_n,x) < \epsilon_x$ whenever $n \geq N$. Define a new sequence $(x_m)$ by $x_m = c_{N+m}$. Note that $d(x_m, x) < \epsilon_x$ for all $m$. This implies that the sequence $(x_m)$ is contained in $B(x, \epsilon_x) \cap C$. But $B(x, \epsilon)$ only intersects $C_i$ for $i$ from $1$ to $k$, so $(x_m) \in F$. Since $\lim x_m =x$, we conclude that $x \in F$ or $x$ is a limit point of $F$. Since $F$ is closed, either case gives us $x \in F$. So $x \in C$ and $C$ contains all of its limit points. Therefore, $C$ is closed. 

	\item  Let $(X,d)$ be a metric space. Let $\{O_{\alpha}\}$ be a collection of open subsets of $X$ for $\alpha$ in some indexing set $I$ with the property that given any $x \in X$, there exists an $\epsilon_x > 0$ such that $B(x, \epsilon_x) \subseteq O_{\alpha}$ for all but a finite number of sets $O_{\alpha}$ in $\{O_{\alpha}\}$. Let $O = \bigcap_{\alpha \in I} O_{\alpha}$. We will prove that $O$ is open by showing that $X \setminus O$ is closed. 

For each $\alpha \in I$, let $C_{\alpha} = X \setminus O_{\alpha}$. Note that each $C_{\alpha}$ is a closed set. Consider the ball $B(x, \epsilon_x)$. Now $B(x, \epsilon_x) \subseteq O_{\alpha}$ for all but a finite number of sets $O_{\alpha}$, so $B(x, \epsilon_x)$ intersects at most finitely many of the sets $C_{\alpha}$. By part (a), we can conclude that $C = \bigcup_{\alpha \in I} C_{\alpha}$ is closed. Since 
\[X \setminus O = X \setminus \bigcap_{\alpha \in I} O_{\alpha} = \bigcup_{\alpha \in I} X \setminus O_{\alpha} = C,\]
it follows that $O$ is open in $X$. 

	\ea

\end{comment}



\item For each of the following, answer true if the statement is always true. If the statement is only sometimes true or never true, answer false and provide a concrete example to illustrate that the statement is false. If a statement is true, explain why. 
	\ba
	\item If $x$ is a point in a metric space $X$, then the singleton set $\{x\}$ is closed. 

	\item The only subsets of $\R$ that are both open and closed under the standard metric are $\emptyset$ and $\R$. 
	
	\item If $(X,d)$ is the metric space with $X = \{1,3,5\}$ and $d(x,y) = xy - 1 \pmod{8}$, then the set $\{1,3\}$ is both open and closed in $X$.
	
	\item If $X$ is a metric space and $A \subseteq X$, then $\Int(\overline{A}) = A$. 
	
	\item The boundary of any subset of a metric space $X$ is a closed set.
	
	\item If $A$ is a subset of a metric space $X$, then $A \subseteq A' \cup A^i$ where $A'$ is the set of limit points of $A$ and $A^i$ is the set of isolated points of $A$

	
	\ea

\begin{comment}

\ExerciseSolution

\ba

	\item This statement is true. If $A = X \setminus \{x\}$ and $a \in A$, then $B(a, d(a,x))$ is a subset of $A$. Thus, $A$ is a neighborhood of each of its points and is open. This makes $\{x\}$ closed. 	
	
	\item This statement is true. Suppose $O$ is an open set in $\R$ different from $\emptyset$ and $\R$. We show that $O$ cannot be closed. Since $O \neq \R$, there is a real number $a \in \R \setminus O$. Since $O$ is not empty, there is a real number $b \in O$. Either $b > a$ or $b < a$. Suppose $b > a$ and let $S = \{d(x,a) \mid x \in O, x > a\}$. Then $S$ is bounded below by $0$ and $S \neq \emptyset$ because $b \in O$. Let $g = \inf(S)$. Now we either have $g \in O$ or $g \notin O$. If $g \in O$, the fact that $O$ is open means that there exists $\epsilon > 0$ such that $B(g, \epsilon) \subseteq O$. Since $a \notin O$, it must be the case that $\epsilon \leq d(g,a)$. But then $g' = g - \frac{1}{2}g(g,a) \in O$ and $d(a,g') < d(a,g)$, which contradicts that fact that $g$ is a lower bound for $S$. So $g \notin O$. To show that $O$ is not closed, we demonstrate that $\R \setminus O$ is not open by showing that $g$ is not an interior point of $\R \setminus O$. To the contrary, suppose there is an $\epsilon > 0$ such that $B(g, \epsilon) \subseteq \R \setminus O$. Then $g+\frac{\epsilon}{2}$ is a lower bound for $S$, which contradicts the fact that $g = \inf(S)$. We conclude that there is no proper subset of $\R$ that is both open and closed. 
	
	\item This statement is true. Recall that the distances in $X$ are as shown in the following table
	\begin{center}
	\begin{tabular}{c|ccc}
		&1	&3	&5 \\ \hline
	1	&0	&2	&4 \\
	3	&2	&0	&6 \\
	5	&4	&6	&0 
	\end{tabular}
	\end{center}
	So $\{1,3\} = B(1,3)$, which makes $\{1,3\}$ an open set in $X$. The complement $X \setminus \{1,3\} = \{5\}$ is equal to $B(5,1)$, which is also open. So $\{1,3\}$ is both open and closed in $X$. 
	
	\item This statement is false. Let $X = \R$ with the Euclidean metric and let $A = [0,1]$. Then $\overline{A} = [0,1]$ and $\Int(\overline{A}) = (0,1) \neq A$. 
		
	\item This statement is true. Let $A$ be a subset of $X$. Then $\Bdry(A) = \overline{A} \cap \overline{X \setminus A}$. So $\Bdry(A)$ is the intersection of two closed sets and so is closed. 
	
	\item This statement is true. To prove $A \subseteq A' \cup A^i$ we begin by letting $a \in A$. If $a \in A'$, then we are done, since $a' \subseteq A' \cup A^i$. So assume that $a \notin A'$. That implies that there must be an $\epsilon > 0$ so that the ball $B(x,\epsilon)$ contains no point of $A$ different than $a$. But this makes $a$ an isolated point of $A$ and $a \in A^i$. In either case, $a \in A' \cup A^i$ and so $A \subseteq A' \cup A^i$.

\ea


\end{comment}

\ee
